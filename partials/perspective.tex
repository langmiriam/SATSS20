Die KI ist aus dem Gesundheitssystem nicht mehr wegzudenken. Die Interaktion des Menschen und KI bleibt im Vordergrund und fordert zwangsläufig die perfekte Mensch-Computer-Symbiose. Sie gibt uns den Anstoß, ein KI - System zu entwickeln, welches intelligente Fragen stellt, anstatt zu versuchen, den Arzt zu ersetzen. Außerdem unterstützen Transparenz und Zuverlässigkeit der KI Algorithmen den Einzug in die Medizin. Zum anderen müssen die Fragen der Verantwortlichkeit und Privatsphäre im Gesundheitswesen weiter ausgearbeitet werden. Wenn alle diese Punkte gelöst werden können wird die KI als ein großartiges Instrument zur Rettung von Leben und zur Verbesserung der Lebensqualität jeden Tag dienen können.