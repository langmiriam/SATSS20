Trotz der großen Menge an Vorteilen, welche die KI Technologien im Gesundheitssystem mitbringen, sind die Herausforderungen für die weitere Etablierung mindestens genau so groß. Für das ordnungsgemäße Funktionieren der KI Systeme ist wesentlich, dass die Leistungserbringer im Gesundheitssystem die Daten, die zur Schulung der KI Systeme verwendet werden, auswerten, um das Einschleichen von Verzerrungen zu verhindern \cite{Chapter_14}. In der Diagnostik können diese Verzerrungen zur falschen Befunden führen.\\

Die Überanpassung von Algorithmen ist eine der Ursachen, die die Anwendung der KI in der Medizin eingrenzen. Von einer Überanpassung spricht man, wenn KI Algorithmen, die auf einen bestimmten Datensatz trainiert wurden, nur begrenzt auf andere Datensätze anwendbar sind \cite{AI_where_are_we_now}. Der KI Algorithmus hat dann ausschließlich die statischen Variationen der Trainingsdaten gelernt, anstatt der allgemeinen, zur Problemlösung notwendigen Konzepte.\\ Die Überanpassung eines Algorithmus wird von mehrere Faktoren beeinflusst wie der Größe des Datensatzes, dem Ausmaß der Heterogenität\footfullcite{heterogen} der Daten und der Verteilung der Daten innerhalb des Datensatzes. Z.B. kann ein Modell über angepasst sein, wenn sich die Krankheitshäufigkeit und die Anzahl der neu auftretenden Krankheitsfälle zwischen den Trainigs- und Testsätzen erheblich unterscheiden \cite{AI_where_are_we_now}. Überanpassung kann aber auch auftreten, wenn die Trainings- und Testsätze mit wesentlich unterschiedlichen Parametern oder Geräten erzielt wurden \cite{AI_where_are_we_now}.\\

Einer der größten Kritikpunkte an der Ki-Integration in die Medizin ist, dass sich KI Techniken von außen betrachtet wie eine Black-Box verhalten\cite{The_missing_pieces}. Gemeint ist, dass es nicht möglich ist, nachzuvollziehen, wie Deep Learning Algorithmen zu ihrer Entscheidung gekommen sind \cite{The_missing_pieces}. Zum Beispiel kann ein Arzt, der von einem KI System einen radiologischen Befund gemeldet bekommt, weder sagen, welche Verfahrensmerkmale für die Analyse verwendet wurden, noch wie sie analysiert wurden und warum der Algorithmus zu diesem Ergebnis kam\cite{The_missing_pieces}. Der Mangel an Transparenz hat deshalb die KI in der wissenschaftlichen Gemeinschaft bisher zurückgedrängt, denn oft ist das „Warum“ hinter einer Vorhersage genauso wichtig wie die Vorhersage selbst \cite{The_missing_pieces}.\\

Weitere Problemfelder für die KI im Gesundheitswesen sind Privatsphäre und Informationssicherheit.\cite{Opportunities_challenges_ai_hc}. In jedem Bereich unserer Gesellschaft ist die IT–Sicherheit inzwischen ein wichtiges Thema geworden. Im Gesundheitswesen wird dies besonders kritisch gesehen, weil Softwareanwendungen unmittelbar mit Menschenleben verbunden sind und ein Cyberangriff im ungünstigsten Fall sogar zum Tod von Patienten führen kann \cite{Opportunities_challenges_ai_hc}.\\
Auf dieser Grundlage stellen sich eine Reihe von weiteren Fragen: Welcher Schutz gilt als zuverlässig? Wer bewertet die Zuverlässigkeit? Und nicht zu vergessen: Wer träg die Verantwortung? 
Die Frage der Verantwortlichkeit wird gerade in einem konkreten Fall diskutiert: bis heute wurde die Roboterchirurgie mit 144 Todesfällen in den Vereinigten Staaten in Verbindung gebracht \cite{Chapter_14}. Wer ist verantwortlich? Das Unternehmen, der Entwickler oder der Dienstleister \cite{Chapter_14}? Diese, noch offene Fragen fordern in unserem Rechtssystem, Datenschutz und Arbeitsverträgen ein Umdenken und Änderungen. \\
