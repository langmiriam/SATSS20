Trotz den großen Menge an Vorteilen die die KI Technologien im Gesundheitssystem mitbringen, sind die Herausforderungen für die weitere Etablierung mindestens genau so groß. Für das ordnungsgemäße Funktionieren der Systeme ist wesentlich, dass die Leistungserbringer im Gesundheitssystem die Daten, die zur Schulung der KI Systeme verwendet werden, auswerten,[14] um das Einschleichen von Verzerrungen zu verhindern. Zum Beispiel wird ein System, das zur Erkennung von Polypen anhand von Diagnosedaten der US-Bevölkerung trainiert wurde, beim Einsatz in asiatischen Ländern nicht besonders effektiv sein, da der asiatischer Typ über einen Phänotyp mit anderen Merkmalen verfügt.

Die Überanpassung von Algorithmen ist eine der Ursachen, welche die Anwendung der KI in der Medizin stark eingrenzen. Von einer Überanpassung spricht man, wenn KI Algorithmen, die auf einen bestimmten Datensatz trainiert wurden, nur begrenzt auf andere Datensätze anwendbar sind. Der Algorithmus hat dann lediglich die statischen Variationen der Trainingsdaten gelernt, anstatt der allgemeinen Konzepte, welche zur Problemlösung notwendig sind. Die Überanpassung eines Algorithmus wird von mehrere Faktoren beeinflusst wie der Größe des Datensatzes, dem Ausmaß der Heterogenität der Daten und der Verteilung der Daten innerhalb des Datensatzes. Z.B. kann ein Modell überangepasst sein, wenn sich die Krankheitshäufigkeit und die Anzahl der neu auftretenden Krankheitsfälle zwischen den Trainigs- und Testsätzen erheblich unterscheiden. Überanpassung kann aber auch auftreten, wenn die Trainings- und Testsätze mit wesentlich unterschiedlichen Parametern oder Geräten erzielt wurden.[now]\\

Einer der größten Kritikpunkte an der Integration der KI in der Medizin ist, dass sich KI Systeme von außen betrachtet wie eine Black-Box verhalten.[paces]. Gemeint ist mit diesem Bild, dass es im Fall von Algorithmen mit tiefem Lernprozess nicht möglich ist, nachzuvollziehen, wie sie zu ihrer Entscheidung gekommen sind.[now] Zum Beispiel kann ein Arzt, der von einem KI System einen radiologischen Befund gemeldet bekommt, weder sagen, welche Verfahrensmerkmale für die Analyse verwendet wurden, noch wie sie analysiert wurden und warum der Algorithmus zu diesem Ergebnis kam?[now]. Der Mangel an Transparenz hat die KI in der wissenschaftlichen Gemeinschaft bisher zurückgedrängt, denn oft ist das „Warum“ hinter einer Vorhersage genauso wichtig wie die Vorhersage selbst. [pieces]. 
Nicht einmal die Korrektheit des Algorithmus lässt sich sicher nachweisen. Versuche, das Black-Box-Problem visuell zu lösen, indem ausgewählte KI-Algorithmen mit Hilfe einer Heatmap dargestellt werden, helfen, die Black-Box der KI zu öffnen. Trotzdem ist es weiterhin unumgänglich, den Mensch in die Entscheidung miteinzubeziehen und Interaktionen zwischen dem KI - System und dem Mensch zu ermöglichen.\\

Weitere Problemfelder für die KI im Gesundheitswesen sind Privatsphäre und Informationssicherheit.[ochallofAIinhc1]. In jedem Bereich unserer Gesellschaft ist die IT – Sicherheit inzwischen ein wichtiges Thema geworden. Im Gesundheitswesen wird dies besonders kritisch gesehen, weil Softwareanwendungen unmittelbar mit Menschenleben verbunden sind und ein Cyberangriffe im ungünstigsten Fall sogar zum Tod von Patienten führen können. Das Remote-Hacking eines Kardiosimulators oder das Umschulen eines KI gestützten Diagnose- und Empfehlungssystem könnte theoretisch für einen Massenmord benutzt werden.[ochallofAIinhc1] 
Auf dieser Grundlage stellen sich eine Reihe von weiteren Fragen: Welcher Schutz gilt als zuverlässig? Wer bewertet die Zuverlässigkeit? Und nicht zu vergessen: Wer träg die Verantwortung? 
Die Frage der Verantwortlichkeit wird gerade in einem konkreten Fall diskutiert: bis heute wurde die Roboterchirurgie mit 144 Todesfällen in den Vereinigten Staaten in Verbindung gebracht.[14] Wer ist verantwortlich? Das Unternehmen, der Entwickler oder der Dienstleister? Auch hier setzt sich zwangsläufig die perfekte Mensch-Computer-Symbiose durch. Sie gibt uns den Anstoß, ein KI - System zu entwickeln, welches intelligente Fragen stellt, anstatt zu versuchen, den Arzt zu ersetzen.\\

Noch im Jahr 2015 wurde während des Wirtschaftsforums in Davos die KI als eine mögliche Bedrohung dargestellt. Stephen Hawking hat damals die Befürchtung geäußert, dass die Menschheit von der KI ausgelöscht werden könnte.[aim] Es ist wichtig solche Vorurteile und Ängste in Bezug auf die KI abzubauen und der größten Befürchtung der Menschheit, dass die KI die Kontrolle über unser Leben übernehmen wird, entgegen zu wirken. Wir müssen uns bemühen, ethische Standards zu schaffen, Erfolg- und Effektivitätsmasstäbe zu entwickeln und die KI – Tools quelloffen, benutzerfreundlich und von erwiesenem klinischem Nutzen machen, dann wird der Einsatz der KI einen gesellschaftlichen Nutzen bringen.[aim] 
