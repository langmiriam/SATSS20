\begin{figure}[h]
    \centering
    \includegraphics[width=1.0\textwidth]{images/bspbild.jpg}
    \caption{\label{fig:bsp}
        Beispielbild
        \protect\footfullcite{web_bspbild}
    }
\end{figure}

Abbildungen werden so referenziert: Wie in Abbildung~\ref{fig:bsp} \ldots.\\
Eine neue Seite machen wir mit \lstinline|newpage|. \newpage
Wir referenzieren ein Kapitel mit \nameref{sec:foo}.\\
Quellen aus dem Internet werden mit Fusszeilenkommentaren zitiert.\footfullcite{web_bspbild}\\
Artikel, Bücher, usw.\ werden mit der normalen Zitierweise zitiert.\cite{usa_ai_approval}\\
Eine \gls{abk} oder mehrere \glspl{abk} kann man als Akronym und Glossareintrag schreiben.
Hierfür müssen sowohl attachments/acronyms.tex als 
auch attachments/glossar.tex angepasst werden.\\
Wir untersuchen die \gls{KI} im medizinischen Umfeld.
