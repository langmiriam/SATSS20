Da KI nur in Verbindung mit Software verwendet werden kann, unterliegt sie der Einteilung und Bewertung nach dem Verfahren für Software. Das bedeutet je nach Einsatzgebiet kann eine KI unterstützte medizinische Software mehr oder weniger stark kontrolliert werden.
Auch wenn die KI keine kritische Aufgabe zu erfüllen hat, fällt sie dennoch und die Risiko-Klasse III von medizinischen Produkten. Das liegt daran dass das Anwendungsgebiet von KI in der Medizin meist bedeutet das sie als Klasse II eingestuft wird. Was dann dazu führt das die KI wie ein Klasse III Produkt geprüft werden muss da es kein Produkt der Klasse II gibt, mit welchem man die KI unterstützte Software vergleichen könnte. Das Problem wird mit dem vorher erwähntem "`De Novo"' Weg zur Zulassung gelöst. 
Mit diesem Verfahren wird akzeptiert, dass es kein vergleichbares Produkt gibt, um den Standard für Klasse II Produkte aufrechterhalten zu können. Das Produkt wird dabei vergleichbar mit Klasse III Produkten geprüft, nur das die Standards und Genauigkeit der Prüfung auf das Produkt angepasst werden. Der "`De Novo"' Weg stellt ein großes Potenzial für die allgemeine Nutzung von KI in der Medizin dar, da mit jeder neuen zugelassenen KI es ein neues, vergleichbares Produkt gibt welches für das Verfahren für Klasse II Geräte genutzt werden kann. Dennoch gibt es keine expliziten Standards nur für KI, es werden immer nur Vergleiche gezogen. Es hängt viel davon ab wie hoch der Standard beim erstem zugelassenem medizinischem Produkt mit KI ist. \cite{usa_ai_approval}

Da KI nur in Verbindung mit Software verwendet werden kann, unterliegt sie der Einteilung und Bewertung nach dem Verfahren für Software. Das bedeutet je nach Einsatzgebiet kann eine KI unterstützte medizinische Software mehr oder weniger stark kontrolliert werden. Da es aber nur wenige vergleichbaren Medizinischen Produkte gibt, werden Medizinische Produkte mit KI der Klasse II automatisch als Klasse III klassifiziert. Dies bedeutet einen deutlich höheren Aufwand bei der Zulassung von Medizin Produkten mit KI, bei denen dieser Aufwand teilweise nicht gerechtfertigt werden kann.
Dieses Problem wird mit dem "`De Novo"' Weg gelöst. Das Produkt wird dabei vergleichbar mit Klasse III Produkten geprüft, nur das die Standards und Genauigkeit der Prüfung auf das Produkt angepasst werden. Dabei verzichtet man auf den Vergleich zu einem existierendem Produkt. Der "`De Novo"' Weg stellt ein großes Potenzial für die allgemeine Nutzung von KI in der Medizin dar, da mit jeder neuen zugelassenen KI es ein neues, vergleichbares Produkt gibt welches für das Verfahren für Klasse II Geräte genutzt werden kann.\cite{usa_ai_approval}