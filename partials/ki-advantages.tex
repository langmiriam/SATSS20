Künstliche Intelligenz hilft nicht nur Ärzten, sondern auch Patienten[chall]. In den letzten Jahren ist insbesondere die Beliebtheit von Telemedizin gestiegen. Solche Anwendungen nutzen unterschiedliche Algorithmen um z. B. Informationen von tragbaren Sensoren auf Fitnesstrackern zu sammeln.[chall1] Diese Geräte werden meistens am Handgelenk getragen und versprechen die Optimierung von Wohlbefinden und Gesundheitszustand in der modernen, von Stress geplagten Gesellschaft. Der Benutzer kann beispielsweise seinen Schlafrhythmus beobachten und auswerten lassen oder seine Fitnesszustand anhand von Parametern wie Herzfrequenz oder verbrauchte Kalorien ständig überwachen.[ochallofAIinhc1] Bereits Kinder tragen gerne solche Geräte und nutzen entsprechende Anwendungen um aus Neugier z. B. die Anzahl der täglichen Schritte zu überprüfen. \\

Andere Softwareanwendungen interagieren mit dem Patienten, um eine Verdachtsdiagnose zu stellen.[ochallofAIinhc1] Solche KI - Anwendungen stellen dem Nutzer Fragen über eine vermutete Erkrankung oder ein gesundheitliches Problem und nutzen Spracherkennung oder Texterkennung, um ihm eine mündliche oder schriftliche Antwort zu ermöglichen.
Zu den beliebtesten Anwendungen dieser Art gehören ADA und Your.MG. ADA als Gesundheitshilfe wurde von einem deutschen Startup Unternehmen entwickelt und weltweit in über 130 Länder in 2016 eingeführt. [ochallofAIinhc1] Es kann kostenlos auf jede beliebige mobile Plattform heruntergeladen werden. Nach Installation und Anmeldung gibt der Nutzer seine Beschwerden und Symptome in einen Textdialog ein und bekommt schließlich Erläuterungen und Ratschläge, was er aufgrund seiner Symptome als nächstes unternehmen soll.\\
 
Eine ganz andere  Anwendung von KI ist Telehomecare, eine Alternative zum traditionellen Krankenhausaufenthalt.[cha14] Die häusliche Pflege ist einer der am schnellsten wachsenden Märkte der Welt.[cha14] In der häuslichen Pflege geht es um Patienten, die postoperativ nach einem Krankenhausaufenthalt oder aufgrund von Alter oder Gebrechlichkeit zu Hause gepflegt werden müssen. Die häusliche Pflege hat sich bereits so weit entwickelt, dass auch Palliativpatienten (sterbende Patienten) zu Hause gepflegt werden können. Dies minimiert Infektionsrisiken und senkt die Kosten für die Patienten erheblich. Zudem trägt die Pflege in einer bekannten Umgebung und mit intensivem Kontakt zur Familie stark zu positiven Behandlungsergebnissen bei.
Bei der Umsetzung von Telehomecare senden Warnsysteme, die sich in der Wohnung der Patienten befinden, Benachrichtigungen an ein Telemonitoring Zentrum, nachdem sie Daten über die spezifischen Symptome und Anzeichen des Patienten gesammelt haben.[cha14]. Diese Information werden in den Telemonitoring Zentren durch eine Krankenschwester und einen Arzt ausgewertet. Sie können in Echtzeit auf kritische Daten des Patienten zugreifen, um Komplikationen zu vermeiden, eine Verschlechterung des Zustands des Patienten zu verhindern oder auf ein Notfall zu reagieren.
Die starke Weiterentwicklung und das Wachstum der mobilen Technologien trägt dazu bei, dass die positiven Auswirkungen der KI gestützten häuslichen Pflege (Telehomecare) noch größer sind, wenn sie mit der Verwendung von tragbaren Geräten (Telemedizin) verbunden sind. Auch hier trägt der Patient Sensoren. z.B. auf einem Armband, die spezifische Information sammeln und auf die Smartphones der Überwachungszentren übertragen. [14] Der Patient kann in Echtzeit seine Herzfrequenz überwachen. Im Fall einer psychischen Krankheit hat die Familie die Möglichkeit, den Aufenthaltsort des Betroffenen zu überwachen.\\ 

Roboter in der Pflege interagieren unmittelbar mit den Patienten. Beispielsweise werden Roboter als Begleiter für die ältere Bevölkerung mit stark eingeschränkten kognitiven Fähigkeiten eingesetzt. Die Vorreiter in diesem Bereich sind die Japaner mit ihren Carebots.