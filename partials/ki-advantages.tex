In den letzten Jahren ist die Beliebtheit der Telemedizin stark gestiegen. Einige Telemedizin-Anwendungen sammeln Informationen von tragbaren Sensoren auf Fitnesstrackern. \cite{Opportunities_challenges_ai_hc}. Diese Geräte werden meistens am Handgelenk getragen und versprechen die Optimierung von Wohlbefinden und Gesundheitszustand in der modernen, von Stress geplagten Gesellschaft. Ein Benutzer oder eine Benutzerin kann beispielsweise seinen Schlafrhythmus analysieren oder seine Fitnesszustand anhand von Parametern wie Herzfrequenz oder verbrauchten Kalorien ständig überwachen \cite{Opportunities_challenges_ai_hc}. \\

Andere Telemedizin-Anwendungen interagieren mit dem Patienten, um eine Verdachtsdiagnose zu stelle n\cite{Opportunities_challenges_ai_hc}. Sie stellen den Nutzer Fragen über eine vermutete Erkrankung oder ein gesundheitliches Problem und nutzen Spracherkennung oder Texterkennung, um ihm eine mündliche oder schriftliche Antwort zu ermöglichen.
Zu den beliebtesten Anwendungen dieser Art gehören ADA und Your.MG. Die Gesundheitshilfe ADA wurde von einem deutschen Startup Unternehmen entwickelt und weltweit in über 130 Länder (Stand 2016) eingeführt \cite{Opportunities_challenges_ai_hc}. ADA kann kostenlos auf alle Mobilgeräte geladen werden. Nach Installation und Anmeldung gibt der Nutzer seine Beschwerden und Symptome in einen Textdialog ein und bekommt Erläuterungen und Ratschläge, was er aufgrund seiner Symptome als nächstes unternehmen soll.\\
 
Eine weiter Anwendung von KI ist Telehomecare, eine Alternative zum traditionellen Krankenhausaufenthalt \cite{Chapter_14}. Die häusliche Pflege ist einer der am schnellsten wachsenden Märkte der Welt \cite{Chapter_14}. Hier geht es um Patienten, die postoperativ nach einem Krankenhausaufenthalt oder aufgrund von Alter oder Gebrechlichkeit zu Hause gepflegt werden müssen. Die häusliche Pflege hat sich bereits so weit entwickelt, dass auch Palliativpatienten zu Hause gepflegt werden können. Palliativepflege ist nach WHO die aktive um­fassen­de Pflege von Patienten, die an ei­ner nicht heil­baren, fort­schreiten­den und so weit fort­ge­schritte­nen Erkrankung leiden, dass dadurch ih­re Le­bens­er­wartung be­grenzt ist \cite{Pshyrembel}. Die Kontrolle des Schmerzes und an­derer Symptome so­wie die Berück­sichtigung psy­cho­lo­gischer, sozialer und spiri­tu­el­ler Belange haben hohe Prio­rität.Dies minimiert Infektionsrisiken und senkt die Kosten für die Patienten erheblich. Zudem trägt die Pflege in einer bekannten Umgebung mit intensivem Kontakt zur Familie stark zu positiven Behandlungsergebnissen bei.
Bei der Umsetzung von Telehomecare senden Warnsysteme, die sich in der Wohnung der Patienten befinden, Benachrichtigungen an ein Telemonitoring Zentrum, nachdem sie Daten über die Bedürfnisse des Patienten gesammelt haben.\cite{Chapter_14} Diese Information werden in den Telemonitoring Zentren durch eine Pfleger/in und einen Arzt/Ärztin ausgewertet. Sie können in Echtzeit auf kritische Daten des Patienten zugreifen, um Komplikationen zu vermeiden, eine Zustandsverschlechterung des Patienten zu verhindern oder auf einen Notfall zu reagieren \cite{Chapter_14}.\\
Die starke Weiterentwicklung und das Wachstum der mobilen Technologien trägt dazu bei, dass die positiven Auswirkungen der KI gestützten häuslichen Pflege (Telehomecare) noch größer sind, wenn sie mit der Verwendung von tragbaren Geräten verbunden sind. Auch hier trägt der Patient Sensoren auf einem Armband, mit denen er in Echtzeit überwacht wird. Im Fall einer psychischen Krankheit hat die Familie die Möglichkeit, den Aufenthaltsort des Betroffenen zu überwachen \cite{Chapter_14}.