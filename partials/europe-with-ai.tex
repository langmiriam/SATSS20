Künstliche Intelligenz (KI) ist ein Teilgebiet der Informatik. Automatisierung von intelligentem Verhalten und maschinelles Lernen (ML) stehen im Vordergrund. Mit dem Einsatz von KI und ML können laufend neue Erkenntnisse gewonnen werden.
Diese sind nützlich, um brauchbare Systeme für Patienten zu entwickeln.\cite{AI_in_EU}\\
In Kapitel~\ref{sec:ki-today} wird auf die Verwendung der Medizinprodukte mit KI eingegangen.
Der Unterschied zwischen Software als Medizinprodukt (SaMD) und Künstlicher Intelligenz mit maschinellen Lerntechnologien liegt darin, dass letzteres die Fähigkeit besitzt, 
Geräteleistungen in Echtzeit anzupassen und zu optimieren.\cite{AI_in_EU}\\
So kann die Gesundheitsversorgung der Patienten durchgehend verbessert werden. 
Obwohl sich die Medizinprodukteverordnung (MDR) nicht ausdrücklich mit medizinischen KI\-Systemen befasst, 
liegt es nahe, dass dieselben Richtlinien auch für KI-Systeme gelten. 
Bei Gesetzen und Normen für medizinische Produkte mit KI wird sich ebenfalls stark an die US\-amerikanische Food and Drug Administration (FDA) gelehnt. 
Dennoch gibt es Unsicherheiten, wie mit solchen Systemen im weiteren Verlauf umgegangen werden soll.\footfullcite{fda} Rahmenbedingungen, die es schon teilweise gibt, sollen den laufenden Entwicklungen angepasst werden.