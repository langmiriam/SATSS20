Künstliche Intelligenz (KI) ist ein Teilgebiet der Informatik.
Automatisierung von intelligentem Verhalten und ML stehen im Vordergrund\cite{AI_in_EU}.
Mit dem Einsatz von KI und ML können laufend neue Erkenntnisse gewonnen werden,
welche nützlich sind, brauchbare Systeme für Patienten zu entwickeln\cite{AI_in_EU}.
In Kapitel~\ref{sec:ki-today} wird auf die Verwendung der Medizinprodukte mit KI eingegangen.
Der Unterschied zwischen SaMD und KI mit ML liegt darin, dass letzteres die Fähigkeit besitzt, 
Geräteleistungen in Echtzeit anzupassen und zu optimieren\cite{AI_in_EU}.
Dadurch kann die Gesundheitsversorgung der Patienten durchgehend verbessert werden. 
Obwohl sich die MDR nicht ausdrücklich mit Medizinprodukten mit KI befasst, 
liegt es nahe, dass dieselben Richtlinien und Normen aus Kapitel~\ref{sec:europe-no-ai},
auch für diese gelten. 
Außerdem wird sich ebenfalls, wie in Kapitel~\ref{sec:us-no-ai} beschrieben, an die Verfahren der FDA gerichtet. 
Die EU-Kommission stellte mit einer Expertengruppe einige zusätzliche Anforderungen auf, wie die technische Robustheit und Sicherheit von Medizinprodukten mit KI, das menschliche Handeln, 
die Nichtdiskriminierung und Fairness. 
Diese wurden 2019 von über 350 Organisationen getestet. 
Die bisherigen gesetzlichen Regelungen für Medizinprodukte beinhalten zwar einige der Anforderungen, 
jedoch fehlt es immer wieder an Transparenz oder Rückverfolgbarkeit\cite{whitepaper}.
Es stellt sich die Frage, welche weiteren Kriterien für die Zulassung der Medizinprodukte mit KI benötigt werden?
Die Datenverwaltung und die Transparenz der Medizinprodukte mit KI ist eine der wichtigen Anforderungen.
Hersteller von KI müssen verantwortungsbewusst mit den Daten der Patienten umgehen. 
Daher ist es notwendig, dass Daten vom Hersteller rechtmäßig und transparent gesammelt werden\cite{Lessons_Learned_about_ai}.\\
Außerdem fehlen bisher ethische und rechtliche Prüfungen bei Medizinprodukten mit KI in der Medizin. 
Trotz gegebenen Rahmenbedingungen gibt es bisher keine Regeln, die sich einfach aufstellen lassen.
Im Bezug auf die Ethik sind die Einsatzgebiete der Medizinprodukte mit KI bis dato jedoch nicht kritisch.
Da aber ständige Weiterentwicklungen der Medizinprodukte mit KI dieses Kriterium nicht ausschließen, ist es erforderlich, dies in Betracht zu ziehen.