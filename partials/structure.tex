Im Kapitel 2 werden die aktuellen Zulassungsverfahren für medizinische Produkte ermittelt. Im Abschnitt 2.1 werden die Zulassungsverfahren der Europäischen Union betrachtet. Zuerst wird untersucht wie medizinische Geräte ohne KI zugelassen werden und dann wie das Verfahren sich bei Geräten mit KI ändert. Im Abschnitt 2.2 wird das gleiche für die Zulassungsverfahren in den Vereinigten Staaten von Amerika getan.\linebreak In Kapitel 3 werden Kriterien für die Zulassung von medizinischen Produkten mit KI erarbeitet. Dazu wird im Abschnitt 3.1  betrachtet wie medizinische Produkte mit KI in der heutigen Medizin verwendet werden. In dem Abschnitt 3.2 werden die Vorteile für den Patienten ermittelt welche sich aus dem Einsatz von KI ergeben und im Abschnitt 3.3 welche Grenzen die KI hat. In Kapitel 4 findet die Evaluation der Ergebnisse aus Kapitel 2 und 3 statt. Im Abschnitt 4.1 stellen wir unsere Lösungsansätze vor für die Probleme welche wir in den Zulassungsverfahren gefunden haben. Dann stellen wir im Abschnitt 4.2 die aktuellen Richtlinien für medizinische Produkte mit KI vor um diese dann im Abschnitt 4.3 zu bewerten. Das Kapitel 5 ist unser Fazit. Im Abschnitt 5.1 werden die Ergebnisse aus Kapitel 4 zusammengefasst. Im Abschnitt 5.2 bewerten wir unsere Ergebnisse und bestimmen wie gültig sie sind und ob wir unsere Forschungsfragen korrekt und allgemeingültig beantworten konnten. Der letzte Abschnitt 5.3 enthält den Ausblick unserer Arbeit und beschreibt wie unsere Arbeit helfen könnte die Zulassungsverfahren für medizinische Produkte mit KI zu entwickeln.