Die Künstliche Intelligenz erlebte seit ihrer Geburtsstunde in den 1940er und 1950er Jahren viele Rückschläge und Erfolge.\cite{Chapter_14} Im Jahr 2011 wurde das maschinelle Lernen für KI entdeckt, was zu einem exponentiellen Entwicklungswachstum der KI führte.\cite{Chapter_14} Der Teilbereich der künstlichen Intelligenz, welcher sich am deutlichsten verbesserte, ist das maschinelle Lernen (ML), bei dem das Lernen durch die Verarbeitung großen Mengen an Trainingsdaten erfolgt.\cite{The_missing_pieces} Auch deshalb wird die KI als die neue Elektrizität und Daten als das neue Öl beschrieben.\cite{Chapter_14}\\

Besonders in der Medizin werden die Fähigkeiten der Künstlichen Intelligenz im Allgemeinen geschätzt. Veränderungen durch KI im Gesundheitswesen betreffen vor allem die bildbasierten diagnostischen Ansätze in den Bereichen Radiologie, Dermatologie und Pathologie. Menschen können zwar Muster in Daten erkennen, jedoch ist dies ein mühsamer Prozess. Ärzte können infolge von Überlastung und Zeitmangel sehr leicht Anzeichen übersehen, was die Diagnose in eine falsche Richtung lenkt.\cite{Opportunities_challenges_ai_hc} Die treibende Kraft hinter der KI- gesteuerten Diagnostik ist die große Anzahl an Bildern, die für die KI zur Verfügung stehen. Die Algorithmen könne dadurch ausreichend trainiert werden.\cite{The_missing_pieces}\\

Die Relevanz der KI im Gesundheitswesen wird auch durch die Tatsache belegt, dass große Unternehmen wie IBM und Google auch auf diesem Gebiet entwickeln. IBM Watson bietet ein Frage-Antwort-System für das Gesundheitswesen an. Es nutzt die Sprache, die Hypothesenbildung und das evidenzbasierte Lernen, um das medizinische Personal bei seinen Entscheidungen zu unterstützen.\cite{Opportunities_challenges_ai_hc} Ein Google Unternehmen eröffnete 2016 eine Abteilung DeepMind Health, welche auch auf dem Gebiet der KI Medizin arbeitet. Diese entwickelt eine ähnliche Anwendung wie IBM Watson, die das medizinische Personal noch differenzierter unterstützen soll.\cite{Opportunities_challenges_ai_hc} In eine andere Richtung geht das Bostoner Startup FDNA: es bietet eine Reihe von Anwendungen unter dem Namen Face2Gene an, die Gesichtsanalyse, KI und genomische Erkenntnisse nutzen, um die Diagnose und Behandlung seltener Krankheiten zu unterstützen.\cite{Opportunities_challenges_ai_hc}\\

Medizinische KI Anwendungen adressieren nicht nur die Ärzte. Quentus, gegründet 2012 in den USA, optimiert Entscheidungen in Krankenhäusern in Echtzeit, um Kosten zu senken, Qualität zu verbessern und Erfahrung zu sammeln. Das Ziel dieser Anwendung ist es, die Abläufe in einem Krankenhaus zu optimieren und zu vereinfachen, damit sich das medizinische Personal auf die Patientenversorgung konzentrieren kann. Quentus entwickelt als Plattform, löst die betrieblichen Herausforderungen im gesamten Krankenhaus einschließlich Notaufnahme, perioperative Bereiche und Patientensicherheit. Diese Anwendung ermöglicht somit die Integration von einem Krankenhaus in ein Gesundheitssystem.\cite{Opportunities_challenges_ai_hc}\\
 
In Kombination mit dem Internet können durch KI weitere Vorteile im medizinischen Bereich entstehen. Die Verwendung von Wearables im Gesundheitswesen bedeutet erhöhte Echtzeit-Transparenz in der gesamten Organisation.\cite{Chapter_14} Durch die stetige Generierung von Patientendaten sind Informationen wie Vitalwerte oder Einhaltung einer Medikation (Compliance) bereits vor der Ankunft eines Patienten im Krankenhaus bekannt. KI Anwendungen sind in der Lage, alle diese Daten zu erfassen und zu verarbeiten. So können Leistungserbringer im Gesundheitswesen Engpässe identifizieren, um Wartezeiten der Patienten zu verkürzen, oder durch die Vermeidung unnötiger Tests Kosten senken.\cite{Chapter_14}\\

Mit dem Gedanken, einen Roboter zu entwickeln, hat vor einigen Jahrhunderten bereits Leonardo da Vinci gespielt.\cite{AI_in_medicine} Entwickler der amerikanischen Firma Intuitive Surgical nannten ihr Chirurgisches System „Da Vinci“.\cite{AI_in_medicine} Es handelt sich um einen Roboter, der komplexe Operationen mit einem minimal-invasiven Zugang erleichtert, und von einem Chirurgen von einer Konsole aus gesteuert werden kann. Es wird zum Beispiel bei Operationen der Herzklappen eingesetzt. 
Da Vinci wurde im Jahr 2000 von der Food and Drug Administartion (FDA) zugelassen. Die Zahl der weltweit eingesetzten Geräte liegt heute bei über 5000.\cite{AI_in_medicine} 