Die Künstliche Intelligenz erlebte seit ihrer Geburtsstunde in den 1940er und 1950er Jahren viele Rückschläge und Erfolge \cite{Chapter_14}. Im Jahr 2011 wurde Deep Learning entdeckt, was zu einem exponentiellen Entwicklungswachstum der KI führte  \cite{Chapter_14}. 
Besonders in der Medizin werden die Fähigkeiten künstlich intelligenter Systeme geschätzt. Weil die KI große Datenmengen verarbeiten, vergleichen und analysieren kann, wird das medizinische Personal entlastet. Menschen können zwar Muster in Daten erkennen, jedoch ist dies ein mühsamer Prozess. Ärzte übersehen infolge von Überlastung und Zeitmangel sehr leicht Anzeichen, was die Diagnose in eine falsche Richtung lenkt \cite{Opportunities_challenges_ai_hc}. Die KI kann helfen, indem sie Signale offenlegt, die sonst nicht erkannt werden \cite{Opportunities_challenges_ai_hc}.\\

Die Relevanz von KI im Gesundheitswesen wird auch durch die Tatsache belegt, dass große Unternehmen wie IBM und Google auf diesem Gebiet entwickeln. IBM Watson bietet ein Frage-Antwort-System für das Gesundheitswesen an. Es nutzt Sprache, Hypothesenbildung und evidenzbasiertes Lernen, um das medizinische Personal bei seinen Entscheidungen zu unterstützen \cite{Opportunities_challenges_ai_hc}. Ein Google Unternehmen eröffnete 2016 eine DeepMind Health Abteilung, welche u.a. auf dem Gebiet der KI Medizin arbeitet. Diese entwickelt eine ähnliche Anwendung wie IBM Watson. Die Anwendungen gibt medizinischem Personal im Einsatz Ratschläge und erkennt Veranlagungen für Krankheiten bei den Patienten  \cite{Opportunities_challenges_ai_hc}.\\

Medizinische KI Anwendungen adressieren nicht nur Ärzte, sondern auch das Management von Unternehmen im Gesundheitswesen \cite{Opportunities_challenges_ai_hc}. Quentus, gegründet 2012 in den USA, optimiert Entscheidungen in Krankenhäusern in Echtzeit um Kosten zu senken, Qualität zu verbessern und Erfahrung zu sammeln. Ziel der Anwendung ist, die Abläufe in einem Krankenhaus zu optimieren und zu vereinfachen, damit sich das medizinische Personal auf die Patientenversorgung konzentrieren kann. Quentus entwickelt als Plattform, löst die betrieblichen Herausforderungen im gesamten Krankenhaus einschließlich Notaufnahme, perioperative Bereiche und Patientensicherheit. Diese Anwendung ermöglicht somit die Integration eines Krankenhauses in ein Gesundheitssystem \cite{Opportunities_challenges_ai_hc}.\\
 
Immer häufiger wird KI in Programmen eingesetzt, die eine gesunden Lebensweise unterstützen sollen. Wearables, auf denen solche Programme laufen, können über ihre Sensoren auch Patientendaten sammeln \cite{Opportunities_challenges_ai_hc}.  Durch die stetige Generierung dieser Daten sind Informationen wie Vitalwerte oder Einhaltung einer Medikation bereits vor der Ankunft eines Patienten im Krankenhaus bekannt. KI Anwendungen sind in der Lage, alle diese Daten zu erfassen und zu verarbeiten. So können Leistungserbringer im Gesundheitswesen Engpässe identifizieren, um Wartezeiten der Patienten zu verkürzen, oder durch die Vermeidung unnötiger Tests Kosten senken \cite{Chapter_14}.\\


 