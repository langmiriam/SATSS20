Die aktuellen Richtlinien behandeln KI unterstützte Software zunächst als gewöhnliche Software. KI unterstützte Medizinische Software wird wie eine gewöhnliche Software zugelassen trotz des Black-Box Problems. Da die Richtlinien für Software auch für KI verwendet werden bedeutet dass das KI zuerst nur unter Laborbedingungen getestet wird, mit einem beschränktem Testdatensatz. Dies kann nicht garantieren das die KI immer funktionieren wird. Tatsache ist aber auch das die Personen in den Ämtern sich dieses Problems bewusst sind und nach besten Wissen und Gewissen versuchen werden die Sicherheit zu gewährleisten. Der amerikanische "De Novo" Weg, an den sich auch die Europäische Union orientiert, bietet die Möglichkeit weniger kritische KI im Feld zu testen und somit ein Qualitätsniveau über Zeit aufzubauen. Kritische KI, welche zum Beispiel hauptverantwortlich für das Leben eines Patienten ist kann immer noch mit den strengsten Kriterien und klinischen Test geprüft werden, bevor die allgemeine Nutzung erlaubt wird. Die aktuellen Richtlinien besitzen noch keine Spezialisierung auf KI, erlauben uns aber medizinische Produkte mit KI zu verwenden und Erfahrungen zu sammeln um ein für KI angepasstes Verfahren zu entwickeln, ohne dabei Menschenleben zu gefährden.