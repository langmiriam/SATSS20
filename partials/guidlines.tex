KI unterstützte Medizinische Software wird wie eine gewöhnliche Software zugelassen, obwohl sie nicht wie eine gewöhnliche Software arbeitet. Tatsache ist aber auch das die Personen in den Ämtern sich dieses Problems bewusst sind und nach besten Wissen und Gewissen versuchen werden die Sicherheit zu gewährleisten. Die in Kapitel \ref{sec:europe} und Kapitel \ref{sec:us} beschriebenen Verfahren sind darauf ausgelegt Innovationen zu ermöglichen. Der amerikanische "`De Novo"' Weg, an den sich auch die Europäische Union orientiert, bietet die Möglichkeit weniger kritische KI im Feld zu testen und ermöglicht das sammeln von Erfahrung bei der Zulassung von Medizinischen Produkten mit KI. Kritische KI, welche zum Beispiel hauptverantwortlich für das Leben eines Patienten sein soll, wird immer noch mit den strengsten Kriterien und klinischen Test geprüft bevor die Markteinführung erlaubt wird. Die aktuellen Richtlinien besitzen noch keine Spezialisierung auf KI, ermöglichen aber das Nutzen von neuen Medizinischen Produkten mit KI. Je nach Einsatzgebiet werden angepasste Zulassungsverfahren gewählt. Medizinische Produkte mit KI werden genauso streng behandelt wie Medizinische Produkte ohne KI welche das selbe Risikoniveau haben.