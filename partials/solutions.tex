Sowohl das Blackbox-Problem auch Überanpassung der Algorithmen bilden zusammen Barrieren für die Zulassung der Medizinprodukte mit KI. Die Zulassungsstellen haben Schwierigkeiten zu bestimmen, wie und ob die Algorithmen funktionieren und ob ihre Leistung auch auf andere Datensätze anwendbar sind[now]. Es wurden aber bereits Möglichkeiten entwickelt die zum einen die Black-Box der KI öffnen und zum anderen die Übereinpassung der Algorithmen mit bestimmten Methoden festgestellt oder verhindert werden können. \\

Bei dem Black-Box-Problem wurde in unterschiedlichen Studien eine Heatmap erstellt, um Regionen mit erhöhten Aktivierung des tiefen Lernnetzwerks aufzuzeichnen. Man schließt daraus, dass es sich bei der stark aktivierten Bereichen um die Regionen handelt die für die Bestimmung der Diagnose entscheidend sind.[now] \\
Die Übereinpassung kann überprüft werden in dem die Algorithmen nach dem Training an mehreren verschiedenen Datensätzen getestet werden. Tests werden grafisch mit einer ROC - receiver operating characteristic (Grenzwertoptimierungskurve) dargestellt. Die ROC - Kurve ist eine Methode zur Bewertung und Optimierung von Analysestrategien und stellt bei den Algorithmen visuell die Abhängigkeit der Effizienz mit der Fehlerrate für Genauigkeit dar. Bei einem Algorithmus der an eine Übereinpassung leidet wurde man erwarten, dass seine Genauigkeit gemessen mit AUC, „area under the receiver operator“ bei Datensätzen die nicht den gleichen Ursprung haben wie die Trainingsdaten, deutliche schlechter sind.\\