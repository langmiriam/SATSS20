Sowohl das Blackbox-Problem auch Überanpassung der Algorithmen bilden zusammen Barrieren für die Zulassung der Medizinprodukte mit KI. Die Zulassungsstellen haben Schwierigkeiten zu bestimmen, wie und ob die Algorithmen funktionieren und ob ihre Leistung auch auf andere Datensätze anwendbar sind.\cite{AI_where_are_we_now} Es wurden aber bereits Möglichkeiten entwickelt die zum einen die Black-Box der KI öffnen und zum anderen die Übereinpassung der Algorithmen mit bestimmten Methoden festgestellt oder verhindert werden können.\cite{AI_where_are_we_now} \\

Bei dem Black-Box-Problem wurde in unterschiedlichen Studien eine Heatmap erstellt, um Regionen mit erhöhten Aktivierung des tiefen Lernnetzwerks aufzuzeichnen. Man schließt daraus, dass es sich bei der stark aktivierten Bereichen um die Regionen handelt die für die Bestimmung der Diagnose entscheidend sind.\cite{AI_where_are_we_now}\\
Die Übereinpassung kann überprüft werden in dem die Algorithmen nach dem Training an mehreren verschiedenen Datensätzen getestet werden.\cite{AI_where_are_we_now} Tests werden grafisch mit einer AUC - area under the curve  dargestellt. Bei einem Algorithmus der an eine Übereinpassung leidet wurde man erwarten, dass seine Genauigkeit gemessen mit AUC, bei Datensätzen die nicht den gleichen Ursprung haben wie die Trainingsdaten, deutliche schlechter sind.\cite{AI_where_are_we_now}\\

Die Einbeziehung verschieder Datentypen, die für KI - Modelle notwendig sind, muss sorgfältig vorgenommen werden um häufige Fallstricke zu vermeiden. Eine Herausforderung bei der Integration verschiedener Daten ist die KI-Modellkomplexität, die entweder auf zusätzliche Funktionen oder eine fortgeschrittene Modellarchitektur zurückzuführen ist. Ein Beispiel für eine komplexes KI-Modell ist das Multiview-Lernen, eine Erweiterungsmethode der Modellarchitektur zur Integration verschiedenen Merkmalstypen. Diese Modell wird verwendet z. B. bei fetalen Ultraschallbilder. Zwar sind solche Modelle schwieriger zu trainieren und potenziell anfälliger für Überanpassung aber solange die Herausforderungen im Auge behalten werden, wird die Einbeziehung verschiedener Datentypen für zukünftige medizinische Modelle vorteilhaft.\cite{The_missing_pieces}