Wie in Kapitel~\ref{sec:europe-with-ai} und Kapitel~\ref{sec:us-with-ai} zu sehen,
gibt es noch kein spezielles/spezifisches Gesetz für die Verwendung von Medizinprodukten mit KI.
Die Zulassungsverfahren von Medizinprodukten ohne KI werden bislang für die Zulassung von Medizinprodukten mit KI angewendet.
Für den Einsatz von Medizinprodukten mit KI, in kritischen Bereichen, müssen die Richtlinien angepasst und ergänzt werden.
Daraus entstehen neue Herausforderungen, die zu bewältigen sind.
Kapitel~\ref{sec:europe-with-ai} zeigt außerdem, dass von der EU-Kommission weitere Anforderungen aufgestellt werden.
In den USA werden ebenso zusätzliche Richtlinien aufgestellt.
Themen, wie ethische und rechtliche Regelungen sowie einige Fragen bleiben allerdings noch offen.
Werden die Daten der Patienten richtig verwendet? Kann die KI voreingenommen sein?
Werden Arbeitsplätze der Ärzte gestrichen? 
Wird die Versorgung auf Grund der KI besser oder schlechter? 
Mit gut durchdachten und klar definierten Regeln für die Verwendung von Medizinprodukten mit KI könnten einige
der aufkommenden Fragen und Bedenken geklärt werden.\\
Des Weiteren soll die Transparenz verbessert und exakte Qualitätsanforderungen aufgestellt werden.
Die oben genannten neuen Anforderungen und die Antworten auf die gestellten Fragen sind wichtig, 
um die Zulassungsverfahren für die Verwendung von Medizinprodukten mit KI zu erweitern,
auszubauen und präziser zu spezifizieren.