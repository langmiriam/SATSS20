KI--Systeme können klinische Entscheidungen ohne direkte Kontrolle von Ärzten übernehmen.\cite{Lessons_Learned_about_ai}
Um KI in der Medizin ohne Bedenken einsetzen zu können, gibt es noch einige Herausforderungen, die zu bewerkstelligen sind. 
Da sich medizinische KI--Systeme ständig weiterentwickeln und immer neue Fortschritte machen, 
müssen die Gesetzgebungen laufend ergänzt werden. 
Zum Teil gibt es Rahmenbedingungen, nach denen vorgegangen werden kann. 
Jedoch ist es keine Pflicht, sich daran zu halten.
In Europa gibt es noch kein spezielles Gesetz für die Verwendung von medizinischen KI--Produkten. 
Wie in Kapitel~\ref{sec:europe-with-ai} erläutert, 
hält man sich stark an die Gesetze und Normen der Zulassung von Medizinprodukten ohne KI.\\ 
Hier sollten jedoch weitere Themen in Betracht gezogen werden. Ethische und rechtliche Regelungen, Datenverwaltung sowie einige Fragen die offen stehen, 
spielen zum Beispiel eine wichtige Rolle. 
Werden die Daten der Patienten richtig verwendet? Kann die KI voreingenommen sein?
Werden Arbeitsplätze der Ärzte gestrichen? 
Wird die Versorgung auf Grund der KI besser oder schlechter? 
Mit gut durchdachten und klar definierten Regeln für die Verwendung von KI--Produkten könnten einige
der aufkommenden Fragen und Bedenken geklärt werden.\\
Bisher fehlen ethische und rechtliche Prüfungen bei KI--Produkten in der Medizin. 
Trotz gegebenen Rahmenbedingungen gibt es bisher keine Regeln, die sich einfach aufstellen lassen. 
Ein weiteres wichtiges Kriterium ist die Datenverwaltung. 
KI--Hersteller müssen verantwortungsbewusst mit den Daten der Patienten umgehen. 
Daher ist es wichtig, dass Daten vom Hersteller rechtmäßig und transparent gesammelt werden.\cite{Lessons_Learned_about_ai}\\
Die EU--Kommission stellte mit einer Expertengruppe einige zusätzliche Anforderungen auf. 
Die technische Robustheit und Sicherheit einer KI--Anwendung, das menschliche Handeln, 
die Datenverwaltung und Transparenz, die Nichtdiskriminierung und Fairness, 
um nur ein paar genannt zu haben. 
Diese wurden 2019 von über 350 Organisationen getestet. 
Die bisherigen gesetzlichen Regelungen beinhalten einige der Anforderungen, 
jedoch fehlt es öfter an Transparenz oder Rückverfolgbarkeit.\cite{whitepaper}\\
In den USA werden ebenso zusätzliche Richtlinien aufgestellt.
Es wird von den Herstellern erwartet, 
die KI--Systeme zu überwachen und Risikomanagement--Ansätze zu erstellen. 
Diese Ansätze beinhalten zum Beispiel Entscheidungen, 
wann eine Softwareänderung vorgenommen werden soll. 
Des Weiteren soll es eine Verbesserung der Transparenz geben 
und exakte Qualitätsanforderungen aufgestellt werden. Eine Überprüfung der SaMD,
bevor sie im Markt zugelassen werden, ist notwendig, 
um die Sicherheit und Wirksamkeit der KI--basierten SaMD zu garantieren.\cite{fda_original}\\
Wie hier deutlich zu sehen ist,
reichen die momentanen Richtlinien und Gesetzgebungen nicht aus. 
Die oben genannten neuen Anforderungen sind wichtig, 
um die Zulassungsverfahren für die Verwendung von Medizinprodukten mit KI zu erweitern,
auszubauen und präziser zu spezifizieren.