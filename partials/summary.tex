Abschließend kann man sagen, dass sich die Zulassung von Medizinprodukten mit KI stark an die Zulassung von Medizinprodukten ohne KI anlehnt. Dies bedeutet jedoch keinesfalls einen reibungslosen Verlauf des Zulassungsverfahren. 
Bei der Zulassung medizinischer Produkte mit KI gestützter Software bestehen Probleme durch mangelnde Vergleichbarkeit der Produkte, fehlende Transparenz oder Überanpassung von KI Algorithmen, undefinierten Verantwortlichkeiten und dem Fehlen von ethischen und rechtlichen Prüfungen. Dies lässt die Zulassungsverfahren langwierig und kostspielig werden.
Für einige dieser Probleme existieren bereits Lösungen. So kann dem Black-Box Problem durch grafische Darstellung diagnoserelevanter Regionen innerhalb des Neuronalen Netzes, dem Problem der Überanpassung durch Vergleiche mit Daten unterschiedlicher Herkunft entgegengewirkt werden.\\ 
Da bisher nur wenige Produkte mit KI auf dem Markt sind, werden, mangels Vergleichbarkeit, Medizinprodukte der Klasse II in die Risikoklasse III eingestuft. Damit diese Zulassungsverfahren nicht mit aufwändigen klinischen Test verbunden sind, gibt es in den USA das beschleunigte De Novo Zulassungsverfahren.\\
Auf die Entwicklung von KI-Systemen für die Medizin reagiert auch die europäische Gesetzgebung. Die EU-Kommission stellt zusätzliche Anforderungen an Transparenz, Robustheit und Sicherheit von Medizinprodukten mit KI. Dies führt jedoch nicht zwangsläufig zur Vereinfachung des Zulassungsverfahren. 
Es steht fest, dass aufgrund der noch offenen Fragen der  Verantwortlichkeiten und dem Fehlen von ethischen und rechtlichen Prüfungen die Zulassungsverfahren um weitere spezifische Kriterien ergänzt werden müssen. 
Es bleibt abzuwarten, ob weitere Maßnahmen seitens der Gesetzgebung einen positiven Einfluss auf die Machbarkeit der Zulassungsverfahren bewirken.


