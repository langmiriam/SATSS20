Der Einsatz von KI in Medizinprodukten stellt die Zulassungsbehörden vor mehrere Herausforderungen. Für das Black-Box-Problem sowie die Überanpassung von KI Algorithmen gibt es bereits Lösungsansätze, welche die Transparenz der Algorithmen verbessern. Diese ist für die Behörden von enormen Wichtigkeit, da es für KI gestützte Software keine spezifischen Zulassungsverfahren gibt. Die Software wird unter Laborbedingungen mit beschränktem Datensatz getestet. Ein erweitertes Zulassungsverfahren unter strengsten Kriterien inklusive klinischer Tests wird nur für Medizinprodukte der Klasse III durchlaufen, da diese bei Nichteinhaltung der Zulassungsanforderungen ein hohes Gefährdungspotential für Patienten besitzen.
Da die Entwicklung von KI Systemen für die Medizin stetig voranschreitet, muss die Gesetzgebung die Zulassungsverfahren fortwährend daran anpassen. Bei der Evaluation der neuen Zulassungskriterien ist es notwendig, dass auch ethische und rechtliche Regelungen in Betracht gezogen werden. Auch Datenschutz spielt bei neuen Zulassungsverfahren eine wichtige Rolle. 
In Europa gibt es aber bei neuen Zulassungskriterien für Medizinprodukte mit KI neuen Ansätze. Die EU-Kommision beschloss mit einer Expertengruppe neue, zusätzliche Anforderungen an Medizinprodukte mit KI. Sie sollen die Transparenz, Robustheit und Sicherheit der KI Anwendungen fordern. Es steht fest, dass die momentane Richtlinien und Gesetzgebung nicht ausreichen und deshalb die Zulassungsverfahren um weitere spezifische Kriterien ergänzt werden müssen.