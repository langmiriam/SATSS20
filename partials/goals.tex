Unsere Hauptforschungsfrage lautet "`Welche Probleme bestehen bei der Zulassung von medizinischen Produkten mit KI unterstützter Software?"'(FF1).
Der Fokus der Arbeit liegt auf den Zulassungsverfahren von medizinischen Produkten. Wir wollen die Zulassungsverfahren für medizinische Produkte untersuchen und prüfen ob sie auch für KI im medizinischem Bereich geeignet sind. Dazu müssen wir zuerst bestimmen welche Kriterien ein Medizinisches Produkt mit KI erfüllen sollte. Deshalb lautet die erste unserer Unterforschungsfragen "`Welche besonderen Kriterien sind für die Zulassung von KI in der Medizin notwendig?"'(FF2). 
Da wir mit unserer Arbeit prüfen wollen ob die Richtlinien für medizinische Produkte mit KI ausreichend sind müssen wir auch den aktuellen Stand der Verfahren untersuchen und was bisher getan wurde um die Richtlinien anzupassen. Deshalb stellten wir uns die Unterforschungsfrage "`Inwieweit gelten die aktuellen Richtlinien auch für Produkte mit KI?"'(FF3). Zuletzt wollen wir untersuchen ob der Einsatz von KI in der Medizin vorteilhaft für den Patienten ist und wie weit diese Vorteile reichen. Dazu formulierten wir als letzte Unterforschungsfrage "`Welche Vorteile bringen KI dem Patienten gegenüber und welche Grenzen haben diese Vorteile?"'(FF4).