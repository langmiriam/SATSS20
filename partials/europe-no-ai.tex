Medizinprodukte wie Röntgengeräte, Implantate, Sehhilfen, Herzschrittmacher, 
Infusionen und Software, sind alles Produkte, die einen bestimmten medizinischen Nutzen am Menschen haben.\footfullcite{medizinprodukte}
Es wird nicht zwischen physikalischen Geräten mit eingebetteter Software und Geräten, die selbst die Software sind, unterschieden. 
Software mit medizinischer Zweckbestimmung wird ebenfalls mit denselben Vorschriften, Richtlinien und Gesetzen entwickelt sowie medizinische Geräte selbst.\cite{AI_in_EU}\\
Medizinprodukte werden mit der Klassifizierungsregel in vier Risikoklassen eingeteilt. 
Die Regeln für die Anwendung der Klassen I, IIa, IIb und III, 
richten sich nach den Zweckbestimmungen der Produkte und liegen in der Verantwortung der Hersteller.\footfullcite{Risikoklassen} Hier werden strenge Anforderungen an die Medizinprodukthersteller verlangt. 
Medizin der Klasse I ist allerdings davon ausgeschlossen, hier reicht eine Selbsterklärung des Herstellers.\footfullcite{MDR}\\
Sobald ein Medizinprodukt vermarktet werden soll, 
muss auf die Medizinprodukteverordnung 2017/745 (MDR), 
die am 25. Mai 2017 in Kraft trat, zurückgegriffen werden.\footfullcite{medizinverordnung} 
Diese stellt die Ansprüche an die Konformitätsbewertungen von Medizinprodukten. 
Die Hersteller medizinischer Software müssen über ein Qualitäts-- und Risikomanagementsystem verfügen, welche für qualitativ hochwertige Produkte sorgen, 
diese aufrecht erhalten und eventuelle Risiken einschätzen können. 
Technische Dokumentationen müssen erstellt und klinische Bewertungen durchgeführt werden. 
Es gibt weiterhin ein zusätzliches nationales Medizinproduktgesetz, 
welches dazu dient, individuellen Regelungen nachzugehen.\cite{AI_in_EU}\\
Laufende Richtlinien werden immer wieder mit neuen Verordnungen korrigiert. 
Der Übergang der Änderungen ist aber ein schwieriger Prozess und nimmt viel Zeit in Anspruch.
Normungsgremien wie International Organization for Standardization (ISO) und International Electrotechnical Commission (IEC) sowie europäische Normungsorganisationen billigen europäische Normen. 
Harmonisierte Normen, die durch die Beantragung der EU-Kommission auf Grund des Harmonisierungsgesetzes veranlasst wurden, 
sind europäische Normen. Diese beinhalten die benötigten rechtlichen Anforderungen. Den Herstellern steht es frei, 
ob sie sich an den harmonisierten Normen orientieren oder nicht. Durch Einhaltung der Normen kann die Konformität nachgewiesen werden.\cite{AI_in_EU}\\
Einige Normen ISO und IEC treffen auf die regulierenden Anforderungen zu.
So werden die Qualitätsanforderungen für die Entwicklung von Medizinprodukten weitestgehend durch die harmonisierte Norm ISO 13485 bestimmt. 
Um für die Sicherheit der Menschen bei klinischer Prüfung von Medizinprodukten zu sorgen, dient die Norm ISO 14155.
Die Norm ISO 14971 ist für den Risikomanagementprozess von Medizinprodukten verantwortlich.
Außerdem folgende drei IEC Normen.
Der Software Lebenszyklus von Medizinprodukten fällt unter die Norm IEC 62304. 
Anwendung der Gebrauchstauglichkeit, durch Entwicklungsprozesse,
auf Medizinprodukte lässt sich auf die Norm IEC 62366--1 zurückführen. 
Die Norm IEC 82304--1 beschäftigt sich mit allgemeinen Anforderungen für die Produktsicherheit.\footfullcite{Normen}
Bei Markteinführung müssen Medizinprodukte alle rechtlichen Anforderungen erfüllen.\\ 
Laut Hersteller ist dies der Fall, mit Verwendung der CE--Kennzeichnung.\cite{AI_in_EU}\\
Des Weiteren werden bei dem Zulassungsprozess Benannte Stellen benötigt. 
Diese sind unabhängige Prüfstellen, die von einem EU--Mitgliedstaat benannt werden und die festgelegte Anforderungen der Produkte der Hersteller begleiten.\footfullcite{Benannte_Stelle} Software entwickelt sich ständig weiter, 
wird verbessert und erneuert. 
Diese Design-Änderungen müssen laufend von der Benannten Stelle überprüft und genehmigt werden, sofern vorgeschriebene Anforderungen beeinträchtigt werden. 
Notified Body Operations (NBOG) bietet Leitfäden für die Benannte Stellen, diese sind nicht rechtsverbindlich, werden aber dennoch verwendet.\cite{AI_in_EU}
