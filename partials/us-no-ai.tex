Die Zulassung von Medizinischen Produkten jeglicher Art fällt unter die Verantwortung der FDA, der Lebensmittelüberwachungs- und Arzneimittelbehörde  der Vereinigten Staaten. Sie entscheiden über die Zulassung von Medikamenten, Lebensmitteln und Medizinischen Geräten sowie auch Medizinischer Software. Autorität zur Überwachung des Marketings und Verkaufs von Pharmazie Produkten oder Medizinischen Produkten bekam die FDA 1938 mit dem "`Federal Food, Drug, and Cosmetic Act"'. Damit war die legislative Basis gegeben. Im Jahr 1970 wurden neue Standards und Einteilungen für Medizinische Geräte präsentiert, da in den Jahren davor sich legislative Lücken gezeigt haben und der Verbraucherschutz nicht gegeben war. Unter anderem war eine Einteilung der Medizinischen Geräte in drei Klassen vorgesehen. Die Klasse I besteht aus Medizinische Produkte ohne ein besonderes Risiko. Die Sicherheit kann mit allgemeinen Kontrollen gewährleistet werden wie bestimmten Produktionsstandards und korrekter Beschriftung.\footfullcite{fdagc} Die Klasse II Produkte haben ein gemäßigtes Risiko und müssen bestimmte Performance Charakteristiken erfüllen um zugelassen zu werden. Dazu werden sie jeweils mit einem vergleichbaren Produkt verglichen um die Sicherheit zu gewährleisten. Außerdem werden sie nach Einführung strenger überwacht und müssen ebenfalls korrekt Beschriftet werden. Klasse III sind die Produkte mit dem höchstem Risiko und müssen vor Markteinführung kontrolliert und in klinischen Tests getestet werden. Außerdem gilt, wenn es für ein Klasse II Produkt kein vergleichbares Produkt gibt, muss es denn gleichen Prozess durchlaufen wie Produkte der Klasse III.\footfullcite{fdacls} Im Jahr 1976 wurden diese Empfehlungen dann umgesetzt. Damit ergibt sich aber das Problem das Klasse II Geräte mit niedrigem Risiko trotzdem die langwierigen Verfahren von Klasse III Produkten durchlaufen müssen. Um das Problem zu lösen wurde der "`De Novo"'\cite{usa_ai_approval} weg zur Zulassung erstellt. Für Geräte mit niedrigem Risiko welche aber keine vergleichbaren Produkte haben. Außerdem gibt es auch für Klasse III Produkte ein beschleunigtes Verfahren wenn es das Potenzial hat Menschen mit schweren und seltenen Krankheiten zu helfen. Die Klasse von Medizinischen Geräten wird bestimmt indem die Datenbank der FDA nach dem Produkttyp durchsucht wird.\footfullcite{fdamdc} Software wird anders gehandhabt. Sie wird nach dem Benutzungsziel klassifiziert. Dabei gibt es vier Klassen mit steigendem Risiko. Klasse I und II beschreibt Software welche nur zur Verwaltung oder zum Informieren des Personals gedacht ist. Klasse II und III sind Klassen für Software welche für kritische Verwaltende Aufgaben genutzt wird oder zur Behandlung und zum Diagnostizieren von Krankheiten verwendet wird. Dabei die höchste Klasse IV für Kritische Einsatzfelder beim Behandeln oder Diagnostizieren reserviert ist. Es gibt noch die Besonderheit das Software, welche ein Gerät steuert nicht in die Software-Klassen eingeteilt wird sondern wie ein normales Gerät behandelt wird. \cite{usa_ai_approval}