{\centering
{\large Abstract (Evelyn)\\}
}
Künstliche Intelligenz (KI) ist ein immer größer werdender, 
stark wachsender Bereich in der Medizin. Sie unterstützt, 
mit der Funktion des maschinellen Lernens, unter anderem das Diagnostizieren von Krankheiten, 
die Entwicklung von Medikamenten und das Personalisieren von Behandlungen. 
Damit medizinische Produkte mit KI in der Medizin eingesetzt werden dürfen, 
gibt es erforderliche Zulassungsverfahren und Vorgaben. 
Diese sind zum Beispiel das Qualitäts- und Risikomanagementsystem, die Datenverwaltung und die Cybersicherheit. 
Welche Verfahren und Ansätze zur Verbesserung in Europa und den Vereinigten Staaten bisher bestehen, 
wird in diesem Artikel zusammen mit den notwendigen Kriterien für die Zulassung von medizinischen KI-Produkten erörtert. 
Die Verwendung dieser Produkte, 
deren Vorteile, Grenzen und weitere Lösungsansätze für die Zulassung werden hier mit einer Literaturanalyse untersucht.
Zudem stellt sich die Frage, 
ob die bisher bestehenden Kriterien für die Zulassung ausreichend sind. 
Trotz bereits angepasster Zulassungsverfahren der medizinischen KI-Produkte, sind weitere Anpassungen der Verfahren durch die regulierenden Behörden nötig.
Dies zeigt sich dadurch, dass kontinuierlich neue Entwicklungen und Verbesserungen der KI in der Medizin hinzukommen. Darüber hinaus liegen keine einheitlichen Gesetzgebungen vor.
