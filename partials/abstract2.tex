{\centering
{\large Abstract (Evelyn)\\}
}
Künstliche Intelligenz (KI) ist ein immer größer werdender, stark wachsender Bereich in der Medizin.
Sie unterstützt, mit der Funktion des maschinellen Lernens,
unter anderem die Diagnosen von Krankheiten, die Entwicklung von Medikamenten,
das Personalisieren von Behandlungen und Verbesserungen von Genbearbeitung.
Damit die medizinischen KI-Produkte in der Medizin eingesetzt werden dürfen,
gibt es erforderliche Zulassungsverfahren und Vorgaben, die eingehalten werden müssen.
Welche Verfahren in Europa und den Vereinigten Staaten bisher bestehen,
wird in diesem Artikel zusammen mit den notwendigen Kriterien
für die Zulassung von medizinischen KI-Produkten erörtert.
Die Verwendung dieser Produkte, deren Vorteile,
Grenzen und weitere Lösungsansätze für die Zulassung sowie die Frage,
ob die bisher bestehenden Kriterien ausreichend sind,
werden hier mit einer Literaturanalyse untersucht.
Trotz bereits angepasster Zulassungsverfahren der medizinischen KI-Produkte,
zeigt sich,
dass weitere Anpassungen durch die regulierenden Behörden nötig sind.