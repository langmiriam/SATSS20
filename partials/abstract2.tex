{\centering
{\large Abstract (Evelyn)\\}
}
Künstliche Intelligenz (KI) ist ein immer größer werdender, 
stark wachsender Bereich in der Medizin.
Sie unterstützt, mit der Funktion des maschinellen Lernens (ML), unter anderem die Diagnose von Krankheiten,
die Entwicklung von Medikamenten und die Personalisierung von Behandlungen. 
Die neue Technologie der KI führt zu notwendigen Anpassungen der Zulassungsverfahren von Medizinprodukten,
da der Weg zu den Ergebnissen der KI nicht vollständig transparent sind.
Damit Medizinprodukte mit KI auf den Markt gebracht werden dürfen,
durchlaufen sie momentan bestehende Zulassungsverfahren von Medizinprodukten.
Welche diese sind und welche Ansätze zur Verbesserung in Europa und den Vereinigten Staaten bisher bestehen, 
wird in diesem Artikel zusammen mit weiteren notwendigen Kriterien für die Zulassung von Medizinprodukten mit KI erörtert. 
Die Verwendung von Medizinprodukten mit KI, 
deren Vorteile, Grenzen und weitere Lösungsansätze für die Zulassung werden hier mit einer Literaturanalyse untersucht.
Zudem stellt sich die Frage, 
ob die bisher bestehenden Kriterien für die Zulassung ausreichend sind. 
Trotz bereits angepasster Zulassungsverfahren der Medizinprodukte mit KI, sind weitere Anpassungen der Verfahren durch die regulierenden Behörden nötig.
Dies zeigt sich dadurch, dass kontinuierlich neue Entwicklungen und Verbesserungen der KI in der Medizin hinzukommen. Darüber hinaus liegen keine einheitlichen Gesetzgebungen vor.

