Um die Ziele der Arbeit genauer zu spezifizieren haben wir mehrere Unterforschungsfragen formuliert. Um zu erfahren ob die aktuellen Zulassungsverfahren für medizinische Produkte auch für medizinische Produkte mit KI geeignet sind müssen wir zuerst definieren was für Kriterien KI im medizinischen Bereich erfüllen müssen, welche bei anderen medizinischen Produkten nicht nötig sind. Dazu die Unterforschungsfrage:
\begin{itemize}
\item Welche besonderen Kriterien sind für die Zulassung von KI in der Medizin notwendig?\end{itemize}
Die nächste Frage welche wir uns gestellt haben ist:\begin{itemize}
\item Inwieweit gelten die aktuellen Richtlinien auch für Produkte mit KI?\end{itemize}
Da wir mit unserer Arbeit prüfen wollen ob die Richtlinien für medizinische Produkte mit KI ausreichend sind müssen wir auch den aktuellen Stand der Verfahren untersuchen und was bisher getan wurde um die Richtlinien anzupassen. Da KI immer ihre Ergebnisse mit einer Wahrscheinlichkeit für deren Richtigkeit versehen müssen wir auch untersuchen ob der Einsatz von KI in der Medizin überhaupt einen Vorteil für den Patienten bringt. Deshalb lautet unsere letzte Unterforschungsfrage:\begin{itemize}\item Welche Vorteile bringen KI dem Patienten gegenüber und welche Grenzen haben diese Vorteile?\end{itemize}