In den letzten Jahren gewinnt Künstliche Intelligenz (KI) immer mehr an Bedeutung. In vielen Bereichen ist die KI nicht wegzudenken. Bei der Analyse von Bildern wird KI weitverbreitet eingesetzt da die klassischen Ansätze der Bildverarbeitung Schwierigkeiten haben auf die verschiedenen Situationen des Alltags zu reagieren. Auch beim Data Mining werden gerne verschiedene Arten von KI eingesetzt um Merkmale und Korrelationen zu erfassen welche nicht normalen statistischen Methoden nicht entdeckt werden können.
Die KI wird seit mehreren Jahren zunehmend auch im Medizinischem Sektor eingesetzt. Dabei stellt die Stärke von KI, die Fähigkeit mit sich ständig ändernden Informationen zu arbeiten, jetzt ein Problem dar. Eine KI kann man nicht mit gewöhnlichen Software-Tests auf Korrektheit prüfen. Das liegt daran dass die Berechnungen im inneren von KI sehr komplex und auf viele Knoten verteilt sind und wir Menschen nicht in der Lage sind diese nachzuvollziehen und zu bewerten. Nun soll KI in einem Gebiet eingesetzt werden welches ein großes Potenzial hat Menschen zu helfen, aber bei Fehlern auch großen Schaden anrichten kann.
Deshalb ist es nötig zu prüfen wie medizinische Produkte mit KI zugelassen werden und ob die Kriterien der Zulassungsverfahren auf die Besonderheiten von KI entsprechend angepasst wurden. Damit wollen wir sicherstellen das die Sicherheit von Patienten, welche mithilfe von Medizinischen Produkten mit KI behandelt werden, garantiert ist.