In den letzten Jahren gewinnt Künstliche Intelligenz (KI) immer mehr an Bedeutung. In vielen Bereichen ist die KI nicht wegzudenken, wie zum Beispiel bei der Bildverarbeitung. Auch im Data Mining ist KI sehr nützlich, weil mithilfe von KI Merkmale und Korrelationen in großen Datenmengen zu erfasst werden können, welche nicht von normalen statistischen Methoden entdeckt werden können. Die KI wird seit mehreren Jahren zunehmend auch im Medizinischem Sektor eingesetzt. Dabei stellt die Stärke von KI, die Fähigkeit mit sich ständig ändernden Informationen zu arbeiten, jetzt ein Problem dar. Aufgrund der Verteilung der Rechnungen auf viele verschiedene Knoten wäre es sehr aufwendig das Neuronale Netz zu verifizieren. Nun soll KI in einem Gebiet eingesetzt werden, welches ein großes Potenzial hat Menschen zu helfen, aber bei Fehlern auch großen Schaden anrichten kann. Deshalb müssen medizinische Produkte strenge Zulassungsverfahren durchlaufen bevor sie eingesetzt werden dürfen. Da diese Verfahren vor dem weitverbreiteten Einsatz von KI erstellt wurden ist es nötig zu prüfen wie medizinische Produkte mit KI zugelassen werden und ob die Kriterien der Zulassungsverfahren auf die Besonderheiten von KI angepasst wurden.