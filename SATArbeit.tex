\documentclass[a4paper, 11pt]{article}
\usepackage[utf8]{inputenc}
\usepackage[T1]{fontenc}
\usepackage{lmodern}
\usepackage[german] {babel}
\usepackage[scaled]{helvet}
\renewcommand\familydefault{\sfdefault} 
\usepackage{lipsum}
\usepackage{cprotect}
\usepackage{fancyhdr}
\usepackage{amsmath}
\usepackage{amsfonts}
\usepackage{amssymb}
%\usepackage{environ}
%\usepackage{makeidx}
\usepackage{tikz}
%\usetikzlibrary{calc,matrix}
%\def\checkmark{\tikz\fill[scale=0.4] (0,.35)-- (.25,0)-- (1,.7)-- (.25,.15)-- cycle;} 
\usepackage{graphicx}
\usepackage{kpfonts}
\usepackage{tabularx}
\usepackage{multirow}
\usepackage{xcolor}
% TIMELINE
\newcommand\ytl[2]{
\parbox[b]{6em}{\hfill{\color{cyan}\bfseries\sffamily #1}~$\cdots\cdots$~}\makebox[0pt][c]{$\bullet$}\vrule\quad \parbox[c]{8.5cm}{\vspace{4pt}\color{black}\raggedright\sffamily #2.\\[5pt]}\\[-3pt]}
\usepackage{booktabs}
\usepackage{graphicx}
\usepackage{footnote}
\makesavenoteenv{figure}
\usepackage{epigraph}
\usepackage{listings}
\lstset{literate=% Allow for German characters in lstlistings.
{Ö}{{\"O}}1
{Ä}{{\"A}}1
{Ü}{{\"U}}1
{ß}{{\ss}}2
{ü}{{\"u}}1
{ä}{{\"a}}1
{ö}{{\"o}}1
}
\lstset{
    language=bash, %% Troque para PHP, C, Java, etc... bash é o padrão
    basicstyle=\ttfamily\small,
    numberstyle=\footnotesize,
    numbers=left,
    backgroundcolor=\color{gray!10},
    frame=single,
    tabsize=2,
    rulecolor=\color{black!30},
    title=\lstname,
    breaklines=true,
    breakatwhitespace=true,
    framextopmargin=2pt,
    framexbottommargin=2pt,
    extendedchars=false,
    inputencoding=utf8
}
\usepackage{hyperref}
\usepackage[
acronym,
nonumberlist,
toc,
nopostdot]{glossaries}
\usepackage[
maxbibnames=99,
backend=biber,
style=numeric,
sorting=ynt
]{biblatex}
\addbibresource{references.bib}
\usepackage{color}
\definecolor{dkgreen}{rgb}{0,0.6,0}
\definecolor{dkblue}{rgb}{0,0,0.4}
\definecolor{gray}{rgb}{0.5,0.5,0.5}
\definecolor{mauve}{rgb}{0.58,0,0.82}
\definecolor{darkerred}{rgb}{0.4,0,0}
\lstset{frame=tb,
    language=Java,
    aboveskip=3mm,
    belowskip=3mm,
    showstringspaces=false,
    columns=flexible,
    basicstyle={\small\ttfamily},
    numbers=none,
    numberstyle=\tiny\color{gray},
    keywordstyle=\color{blue},
    commentstyle=\color{dkgreen},
    stringstyle=\color{mauve},
    breaklines=true,
    breakatwhitespace=true,
    tabsize=3
}
\hypersetup{
    colorlinks = true,
    linkcolor=dkblue,
    filecolor=magenta,      
    urlcolor=cyan,
}
\renewcommand*{\glstextformat}[1]{\textcolor{darkerred}{#1}}
\usepackage[doublespacing]{setspace}
\usepackage{background}
\usepackage{lastpage}
\newcommand{\subautor}[1]{\begin{flushright}
    {\small\textit{#1}}    
\end{flushright}}
\backgroundsetup{contents={}}
\pagestyle{fancy}
\renewcommand{\headrulewidth}{0pt}% removes header line
\lhead{\leftmark}
\rhead{}
\cfoot{\thepage}
\lfoot{\hyperlink{contents}{\small{Inhaltsverzeichnis}}}% links the TOC at the center of the page footer

\AtBeginDocument{\addtocontents{toc}{\protect\thispagestyle{empty}}}

\makeglossaries{}
\input{attachments/glossar-alex}
\newglossaryentry{g-abk}{
    name={Abkürzung},
    plural={Abkürzungen},
    description={
        Das ist nur ein Beispiel für einen Akronym und Glossareinträge.
    }
}

\input{attachments/glossar-miri}
\input{attachments/acronyms-alex}
\newglossaryentry{abk}{
    type=\acronymtype,
    name={Abk.},
    plural={Abkn.},
    description={\gls{g-abk}},
    first={Abkürzung (Abk.)\glsadd{g-abk}},
    firstplural={Abkürzungen (Abkn.)\glsadd{g-abk}},
}
\newglossaryentry{KI}{
	type=\acronymtype,
	name={KI},
    plural={KI},
    description={Künstliche Intelligenz},
    first={Künstliche Intelligenz (KI)},
    firstplural={Künstliche Intelligenzen (KI)},
}
\input{attachments/acronyms-miri}
\begin{document}
    \pagestyle{empty}
    \begin{titlepage}
    \begin{figure}
        \begin{flushright}
            \includegraphics[scale=0.75]{images/INFLogo.png}
        \end{flushright}
    \end{figure}

    \centering
    \vspace{1.5cm}
    {\Large Artefakt}\\

    \vspace{0.5cm}
    {\Large SAT}\\
    {\Large SoSe 2020}\\

    \vspace{1.0cm}
    \Large{\textbf{
            Kriterien für die Zulassung von medizinischen Produkten mit Künstlicher Intelligenz
          }
    }\\

    \vspace{1.0cm}
    \setstretch{0,8}
    {\small Alex Pollok}\\
    {\small 764359}\\
    {\small \href{mailto:alex_mark.pollok@student.reutlingen-university.de}{alex{\textunderscore}mark.pollok@student.reutlingen-university.de}}\\
    \vspace{0,5cm}
    {\small Evelyn Krebes}\\
    {\small 762780}\\
    {\small \href{mailto:evelyn_sophie.krebes@student.reutlingen-university.de}{evelyn{\textunderscore}sophie.krebes@student.reutlingen-university.de}}\\
    \vspace{0,5cm}
    {\small Miriam Lang}\\
    {\small 764532}\\
    {\small \href{mailto:miriam.lang@student.reutlingen-university.de}{miriam.lang@student.reutlingen-university.de}}\\

    \vspace{1,5cm}
    {\small Betreuer: Prof. Dr. rer. nat. Christian Kücherer\\}
    \vspace{1.5cm}
    {\small \input{partials/abstract.tex}}
    \backgroundsetup{
      scale=1,
      color=black,
      opacity=1,
      angle=0,
      position=current page.south,
      vshift=60pt,
      hshift=-210pt,
      contents={%
      \begin{minipage}{.18\textwidth}
      \includegraphics[width=1000pt,height=70pt,keepaspectratio]{images/FHRTFooter.png}
      \end{minipage}%
      }
    }
\end{titlepage}

    \newpage
    \setstretch{1.0}
    {\centering
{\large Abstract (Evelyn)\\}
}
Künstliche Intelligenz (KI) ist ein immer größer werdender, 
stark wachsender Bereich in der Medizin.
Sie unterstützt, mit der Funktion des maschinellen Lernens (ML), unter anderem die Diagnose von Krankheiten,
die Entwicklung von Medikamenten und die Personalisierung von Behandlungen. 
Die neue Technologie der KI führt zu notwendigen Anpassungen der Zulassungsverfahren von Medizinprodukten,
da der Weg zu den Ergebnissen der KI nicht vollständig transparent sind.
Damit Medizinprodukte mit KI auf den Markt gebracht werden dürfen,
durchlaufen sie momentan bestehende Zulassungsverfahren von Medizinprodukten.
Welche diese sind und welche Ansätze zur Verbesserung in Europa und den Vereinigten Staaten bisher bestehen, 
wird in diesem Artikel zusammen mit weiteren notwendigen Kriterien für die Zulassung von Medizinprodukten mit KI erörtert. 
Die Verwendung von Medizinprodukten mit KI, 
deren Vorteile, Grenzen und weitere Lösungsansätze für die Zulassung werden hier mit einer Literaturanalyse untersucht.
Zudem stellt sich die Frage, 
ob die bisher bestehenden Kriterien für die Zulassung ausreichend sind. 
Trotz bereits angepasster Zulassungsverfahren der Medizinprodukte mit KI, sind weitere Anpassungen der Verfahren durch die regulierenden Behörden nötig.
Dies zeigt sich dadurch, dass kontinuierlich neue Entwicklungen und Verbesserungen der KI in der Medizin hinzukommen. Darüber hinaus liegen keine einheitlichen Gesetzgebungen vor.


        
    \newpage
    \hypertarget{contents}{}
    \tableofcontents

    \newpage
    \pagestyle{fancy}
    \setstretch{1.0}
    \section{Einführung \small{(Alex)}}\label{sec:introduction}
        In diesem Kapitel geben zuerst einen Überblick über den Kontext der Arbeit und weshalb die Zulassungskriterien für Medizinische Produkte mit Künstlicher Intelligenz eine genauere Untersuchung benötigen. Dann stellen wir die Ziele unserer Arbeit vor und wie wir unsere Forschungsfragen beantworten wollen. Zuletzt wird die Struktur der Arbeit kurz dargestellt.
		\subsection{Motivation, Kontext und Gegenstand}\label{sec:motivationcontext}
			In den letzten Jahren gewinnt Künstliche Intelligenz (KI) immer mehr an Bedeutung. In vielen Bereichen ist die KI nicht wegzudenken, wie zum Beispiel bei der Bildverarbeitung. Auch im Data Mining ist KI sehr nützlich, weil mithilfe von KI Merkmale und Korrelationen in großen Datenmengen zu erfasst werden können, welche nicht von normalen statistischen Methoden entdeckt werden können. Die KI wird seit mehreren Jahren zunehmend auch im Medizinischem Sektor eingesetzt. Dabei stellt die Stärke von KI, die Fähigkeit mit sich ständig ändernden Informationen zu arbeiten, jetzt ein Problem dar. Aufgrund der Verteilung der Rechnungen auf viele verschiedene Knoten wäre es sehr aufwendig das Neuronale Netz zu verifizieren. Nun soll KI in einem Gebiet eingesetzt werden, welches ein großes Potenzial hat Menschen zu helfen, aber bei Fehlern auch großen Schaden anrichten kann. Deshalb müssen medizinische Produkte strenge Zulassungsverfahren durchlaufen bevor sie eingesetzt werden dürfen. Da diese Verfahren vor dem weitverbreiteten Einsatz von KI erstellt wurden ist es nötig zu prüfen wie medizinische Produkte mit KI zugelassen werden und ob die Kriterien der Zulassungsverfahren auf die Besonderheiten von KI angepasst wurden.
		\subsection{Ziele}\label{sec:goals}
			In diesem Abschnitt werden unsere Hauptforschungsfrage genannt und erklärt um das allgemeine Ziel der Arbeit zu vermitteln. Anschließend werden die Unterforschungsfragen vorgestellt und damit die Ziele der Arbeit genauer eingegrenzt.
		\subsection{Vorgehensweise}\label{sec:procedure}
			Um unsere Forschungsfragen zu beantworten werden wir die aktuellen Richtlinien für die Zulassung von medizinischen Produkten mit KI der Vereinigten Staaten von Amerika untersuchen und auch die Europäischen Richtlinien für solche Produkte. Dies passiert in Form eines Literatur Review. Mit dieser Methode werden wir ermitteln ob und welche besonderen Kriterien für die Zulassung von KI im medizinischem Bereich es gibt. Dazu werden wir nach Beispielen suchen wie KI in der heutigen Medizin Anwendung findet und was für Vorteile und Grenzen die KI hat.\linebreak Anschließend werden wir die Ergebnisse unserer Recherche untersuchen und im Hinblick auf unsere Forschungsfragen mithilfe von Standards aus anderen Gebieten die aktuelle Lage bewerten.  Außerdem stellen wir unsere Lösungsansätze vor, wie man die ermittelten Problem in den Zulassungsverfahren beheben könnte.
		\subsection{Aufbau der Arbeit}\label{sec:structure}
			\begin{itemize}
	\item Literatur Review, Research, Review of Guidelines, Comparison to Industry standardss
\end{itemize}

	\newpage
	\section{Bisherige Zulassungsverfahren von medizinischen Produkten}\label{sec:admission}
		\begin{itemize}
	\item Zeigen wie Geräte verschiedenster Art bisher zugelassen werden
\end{itemize}
		\subsection{Zulassungsverfahren medizinischer Geräte \small{(Evelyn)}}\label{sec:medproducts}
			Um Medizinprodukte in Europa und den USA einsetzen zu dürfen,
müssen sie zuvor entsprechenden Zulassungsverfahren unterzogen werden.
Diese Verfahren beinhalten verschiedene regulatorische Anforderungen und Richtlinien,
welche in Kapitel~\ref{sec:europe-no-ai} und Kapitel~\ref{sec:us-no-ai} näher beschrieben werden. 		
			\subsubsection{Medizinische Geräte in Europa \small{(Evelyn)}}\label{sec:europe-no-ai}
				Medizinprodukte wie Röntgengeräte, Implantate, Sehhilfen, Herzschrittmacher, 
Infusionen und Software, sind alles Produkte, die einen bestimmten medizinischen Nutzen am Menschen haben.\footfullcite{medizinprodukte}
Es wird nicht zwischen physikalischen Geräten mit eingebetteter Software und Geräten, die selbst die Software sind, unterschieden. 
Software mit medizinischer Zweckbestimmung wird ebenfalls mit denselben Vorschriften, Richtlinien und Gesetzen entwickelt sowie medizinische Geräte selbst.\cite{AI_in_EU}\\
Medizinprodukte werden mit der Klassifizierungsregel in vier Risikoklassen eingeteilt. 
Die Regeln für die Anwendung der Klassen I, IIa, IIb und III, 
richten sich nach den Zweckbestimmungen der Produkte und liegen in der Verantwortung der Hersteller.\footfullcite{Risikoklassen} Hier werden strenge Anforderungen an die Medizinprodukthersteller verlangt. 
Medizin der Klasse I ist allerdings davon ausgeschlossen, hier reicht eine Selbsterklärung des Herstellers.\footfullcite{MDR}\\
Sobald ein Medizinprodukt vermarktet werden soll, 
muss auf die Medizinprodukteverordnung 2017/745 (MDR), 
die am 25. Mai 2017 in Kraft trat, zurückgegriffen werden.\footfullcite{medizinverordnung} 
Diese stellt die Ansprüche an die Konformitätsbewertungen von Medizinprodukten. 
Die Hersteller medizinischer Software müssen über ein Qualitäts-- und Risikomanagementsystem verfügen, welche für qualitativ hochwertige Produkte sorgen, 
diese aufrecht erhalten und eventuelle Risiken einschätzen können. 
Technische Dokumentationen müssen erstellt und klinische Bewertungen durchgeführt werden. 
Es gibt weiterhin ein zusätzliches nationales Medizinproduktgesetz, 
welches dazu dient, individuellen Regelungen nachzugehen.\cite{AI_in_EU}\\
Laufende Richtlinien werden immer wieder mit neuen Verordnungen korrigiert. 
Der Übergang der Änderungen ist aber ein schwieriger Prozess und nimmt viel Zeit in Anspruch.
Normungsgremien wie International Organization for Standardization (ISO) und International Electrotechnical Commission (IEC) sowie europäische Normungsorganisationen billigen europäische Normen. 
Harmonisierte Normen, die durch die Beantragung der EU-Kommission auf Grund des Harmonisierungsgesetzes veranlasst wurden, 
sind europäische Normen. Diese beinhalten die benötigten rechtlichen Anforderungen. Den Herstellern steht es frei, 
ob sie sich an den harmonisierten Normen orientieren oder nicht. Durch Einhaltung der Normen kann die Konformität nachgewiesen werden.\cite{AI_in_EU}\\
Einige Normen ISO und IEC treffen auf die regulierenden Anforderungen zu.
So werden die Qualitätsanforderungen für die Entwicklung von Medizinprodukten weitestgehend durch die harmonisierte Norm ISO 13485 bestimmt. 
Um für die Sicherheit der Menschen bei klinischer Prüfung von Medizinprodukten zu sorgen, dient die Norm ISO 14155.
Die Norm ISO 14971 ist für den Risikomanagementprozess von Medizinprodukten verantwortlich.
Außerdem folgende drei IEC Normen.
Der Software Lebenszyklus von Medizinprodukten fällt unter die Norm IEC 62304. 
Anwendung der Gebrauchstauglichkeit, durch Entwicklungsprozesse,
auf Medizinprodukte lässt sich auf die Norm IEC 62366--1 zurückführen. 
Die Norm IEC 82304--1 beschäftigt sich mit allgemeinen Anforderungen für die Produktsicherheit.\footfullcite{Normen}
Bei Markteinführung müssen Medizinprodukte alle rechtlichen Anforderungen erfüllen.\\ 
Laut Hersteller ist dies der Fall, mit Verwendung der CE--Kennzeichnung.\cite{AI_in_EU}\\
Des Weiteren werden bei dem Zulassungsprozess Benannte Stellen benötigt. 
Diese sind unabhängige Prüfstellen, die von einem EU--Mitgliedstaat benannt werden und die festgelegte Anforderungen der Produkte der Hersteller begleiten.\footfullcite{Benannte_Stelle} Software entwickelt sich ständig weiter, 
wird verbessert und erneuert. 
Diese Design-Änderungen müssen laufend von der Benannten Stelle überprüft und genehmigt werden, sofern vorgeschriebene Anforderungen beeinträchtigt werden. 
Notified Body Operations (NBOG) bietet Leitfäden für die Benannte Stellen, diese sind nicht rechtsverbindlich, werden aber dennoch verwendet.\cite{AI_in_EU}

			\subsubsection{Medizinische Geräte in den USA \small{(Alex)}}\label{sec:us-no-ai}
				Die Zulassung von Medizinischen Produkten jeglicher Art fällt unter die Verantwortung der FDA, der Lebensmittelüberwachungs- und Arzneimittelbehörde  der Vereinigten Staaten. Sie entscheiden über die Zulassung von Medikamenten, Lebensmitteln und Medizinischen Geräten sowie auch Medizinischer Software. Autorität zur Überwachung des Marketings und Verkaufs von Pharmazie Produkten oder Medizinischen Produkten bekam die FDA 1938 mit dem "`Federal Food, Drug, and Cosmetic Act"'. Damit war die legislative Basis gegeben. Im Jahr 1970 wurden neue Standards und Einteilungen für Medizinische Geräte präsentiert, da in den Jahren davor sich legislative Lücken gezeigt haben und der Verbraucherschutz nicht gegeben war. Unter anderem war eine Einteilung der Medizinischen Geräte in drei Klassen vorgesehen. Die Klasse I besteht aus Medizinische Produkte ohne ein besonderes Risiko. Die Sicherheit kann mit allgemeinen Kontrollen gewährleistet werden wie bestimmten Produktionsstandards und korrekter Beschriftung.\footfullcite{fdagc} Die Klasse II Produkte haben ein gemäßigtes Risiko und müssen bestimmte Performance Charakteristiken erfüllen um zugelassen zu werden. Dazu werden sie jeweils mit einem vergleichbaren Produkt verglichen um die Sicherheit zu gewährleisten. Außerdem werden sie nach Einführung strenger überwacht und müssen ebenfalls korrekt Beschriftet werden. Klasse III sind die Produkte mit dem höchstem Risiko und müssen vor Markteinführung kontrolliert und in klinischen Tests getestet werden. Außerdem gilt, wenn es für ein Klasse II Produkt kein vergleichbares Produkt gibt, muss es denn gleichen Prozess durchlaufen wie Produkte der Klasse III.\footfullcite{fdacls} Im Jahr 1976 wurden diese Empfehlungen dann umgesetzt. Damit ergibt sich aber das Problem das Klasse II Geräte mit niedrigem Risiko trotzdem die langwierigen Verfahren von Klasse III Produkten durchlaufen müssen. Um das Problem zu lösen wurde der "`De Novo"'\cite{usa_ai_approval} weg zur Zulassung erstellt. Für Geräte mit niedrigem Risiko welche aber keine vergleichbaren Produkte haben. Außerdem gibt es auch für Klasse III Produkte ein beschleunigtes Verfahren wenn es das Potenzial hat Menschen mit schweren und seltenen Krankheiten zu helfen. Die Klasse von Medizinischen Geräten wird bestimmt indem die Datenbank der FDA nach dem Produkttyp durchsucht wird.\footfullcite{fdamdc} Software wird anders gehandhabt. Sie wird nach dem Benutzungsziel klassifiziert. Dabei gibt es vier Klassen mit steigendem Risiko. Klasse I und II beschreibt Software welche nur zur Verwaltung oder zum Informieren des Personals gedacht ist. Klasse II und III sind Klassen für Software welche für kritische Verwaltende Aufgaben genutzt wird oder zur Behandlung und zum Diagnostizieren von Krankheiten verwendet wird. Dabei die höchste Klasse IV für Kritische Einsatzfelder beim Behandeln oder Diagnostizieren reserviert ist. Es gibt noch die Besonderheit das Software, welche ein Gerät steuert nicht in die Software-Klassen eingeteilt wird sondern wie ein normales Gerät behandelt wird. \cite{usa_ai_approval}	
		\subsection{Zulassungsverfahren medizinischer Geräte mit KI \small{()}}\label{sec:medproductsAI}
			\input{partials/medproductsAI}			
			\subsubsection{Medizinische Geräte mit KI in Europa \small{(Evelyn)}}\label{sec:europe-with-ai}
				Künstliche Intelligenz (KI) ist ein Teilgebiet der Informatik. Automatisierung von intelligentem Verhalten und maschinelles Lernen (ML) stehen im Vordergrund. Mit dem Einsatz von KI und ML können laufend neue Erkenntnisse gewonnen werden.
Diese sind nützlich, um brauchbare Systeme für Patienten zu entwickeln.\cite{AI_in_EU}\\
In Kapitel~\ref{sec:ki-today} wird auf die Verwendung der Medizinprodukte mit KI eingegangen.
Der Unterschied zwischen Software als Medizinprodukt (SaMD) und Künstlicher Intelligenz mit maschinellen Lerntechnologien liegt darin, dass letzteres die Fähigkeit besitzt, 
Geräteleistungen in Echtzeit anzupassen und zu optimieren.\cite{AI_in_EU}\\
So kann die Gesundheitsversorgung der Patienten durchgehend verbessert werden. 
Obwohl sich die Medizinprodukteverordnung (MDR) nicht ausdrücklich mit medizinischen KI\-Systemen befasst, 
liegt es nahe, dass dieselben Richtlinien auch für KI-Systeme gelten. 
Bei Gesetzen und Normen für medizinische Produkte mit KI wird sich ebenfalls stark an die US\-amerikanische Food and Drug Administration (FDA) gelehnt. 
Dennoch gibt es Unsicherheiten, wie mit solchen Systemen im weiteren Verlauf umgegangen werden soll.\footfullcite{fda} Rahmenbedingungen, die es schon teilweise gibt, sollen den laufenden Entwicklungen angepasst werden.
			\subsubsection{Medizinische Geräte mit KI in den USA \small{(Alex)}}\label{sec:us-with-ai}
				Da KI nur in Verbindung mit Software verwendet werden kann unterliegt sie der Einteilung und Bewertung nach dem Verfahren für Software. Das bedeutet je nach Einsatzgebiet kann eine KI unterstützte, medizinische Software mehr oder weniger stark kontrolliert werden. Auch wenn die KI keine kritische Aufgabe zu erfüllen hat, fällt sie dennoch und die Risiko-Klasse III von medizinischen Produkten. Das liegt daran dass das Anwendungsgebiet von KI in der Medizin meist bedeutet das sie als Klasse II eingestuft wird. Was dann dazu führt das die KI wie ein Klasse III Produkt geprüft werden muss da es kein Produkt der Klasse II gibt, mit welchem man die KI unterstützte Software vergleichen könnte. Das Problem wird mit dem vorher erwähntem "De Novo" Weg zur Zulassung gelöst. Mit diesem Verfahren wird akzeptiert das es kein vergleichbares Produkt gibt um den Standard für Klasse II Produkte aufrechterhalten zu können. Das Produkt wird dabei vergleichbar mit Klasse III Produkten geprüft, nur das die Standards und Genauigkeit der Prüfung auf das Produkt angepasst werden. Der "De Novo" Weg stellt ein großes Potenzial für die allgemeine Nutzung von KI in der Medizin dar, da mit jeder neuen zugelassenen KI es ein neues, vergleichbares Produkt gibt welches für das Verfahren für Klasse II Geräte genutzt werden kann. Dennoch gibt es keine expliziten Standards nur für KI, es werden immer nur Vergleiche gezogen. Es hängt viel davon ab wie hoch der Standard beim erstem zugelassenem medizinischem Produkt mit KI ist. \cite{YAEGER2019192}
				
				

	\newpage
	\section{Einsatz von KI in der Medizin}\label{sec:analysis}
				\input{partials/special-criteria}
			\subsection{Verwendung medizinischer Produkte mit KI in der heutigen Medizin \small{(Miriam)}}\label{sec:ki-today}
				Die Künstliche Intelligenz erlebte seit ihrer Geburtsstunde in den 1940er und 1950er Jahren viele Rückschläge und Erfolge \cite{Chapter_14}. Im Jahr 2011 wurde das maschinelle Lernen (ML) entdeckt, was zu einem exponentiellen Entwicklungswachstum der KI führte  \cite{Chapter_14}. Besonders in der Medizin werden die Fähigkeiten künstlich intelligenter Systeme geschätzt.Dabei wird die KI im Gesundheitssystem am häufigsten für die Unterstützung bei der Diagnose, im Management und für die Erhaltung einer gesunden Lebensweise eingesetzt \cite{Opportunities_challenges_ai_hc}.\\
Weil die KI große Datenmengen verarbeiten, vergleichen und analysieren kann, wird das medizinische Personal entlastet. Menschen können zwar Muster in Daten erkennen, jedoch ist dies ein mühsamer Prozess. Ärzte übersehen infolge von Überlastung und Zeitmangel sehr leicht Anzeichen, was die Diagnose in eine falsche Richtung lenkt \cite{Opportunities_challenges_ai_hc}. Die KI kann helfen, indem sie Signale offenlegt, die sonst nicht erkannt werden \cite{Opportunities_challenges_ai_hc}.\\
Die Relevanz von KI im Gesundheitswesen wird auch durch die Tatsache belegt, dass große Unternehmen wie IBM und Google auf diesem Gebiet entwickeln. IBM Watson bietet ein Frage-Antwort-System für das Gesundheitswesen an. Es nutzt Sprache, Hypothesenbildung und evidenzbasiertes Lernen, um das medizinische Personal bei seinen Entscheidungen zu unterstützen \cite{Opportunities_challenges_ai_hc}. Ein Google Unternehmen eröffnete 2016 eine DeepMind Health Abteilung, welche u.a. auf dem Gebiet der KI Medizin arbeitet. Diese entwickelt eine ähnliche Anwendung wie IBM Watson. Die Anwendungen gibt medizinischem Personal im Einsatz Ratschläge und erkennt Veranlagungen für Krankheiten bei den Patienten  \cite{Opportunities_challenges_ai_hc}.\\
Medizinische KI Anwendungen adressieren nicht nur Ärzte, sondern auch das Management von Unternehmen im Gesundheitswesen \cite{Opportunities_challenges_ai_hc}. Quentus, gegründet 2012 in den USA, optimiert Entscheidungen in Krankenhäusern in Echtzeit um Kosten zu senken, Qualität zu verbessern und Erfahrung zu sammeln. Ziel der Anwendung ist, die Abläufe in einem Krankenhaus zu optimieren und zu vereinfachen, damit sich das medizinische Personal auf die Patientenversorgung konzentrieren kann. Quentus entwickelt als Plattform, löst die betrieblichen Herausforderungen im gesamten Krankenhaus einschließlich Notaufnahme, perioperative Bereiche und Patientensicherheit. Diese Anwendung ermöglicht somit die Integration eines Krankenhauses in ein Gesundheitssystem \cite{Opportunities_challenges_ai_hc}.\\
Immer häufiger wird KI in Programmen eingesetzt, die eine gesunden Lebensweise unterstützen sollen. Wearables, auf denen solche Programme laufen, können über ihre Sensoren auch Patientendaten sammeln \cite{Opportunities_challenges_ai_hc}.  Durch die stetige Generierung dieser Daten sind Informationen wie Vitalwerte oder Einhaltung einer Medikation bereits vor der Ankunft eines Patienten im Krankenhaus bekannt. KI Anwendungen sind in der Lage, alle diese Daten zu erfassen und zu verarbeiten. So können Leistungserbringer im Gesundheitswesen Engpässe identifizieren, um Wartezeiten der Patienten zu verkürzen, oder durch die Vermeidung unnötiger Tests Kosten senken \cite{Chapter_14}.\\


 
			\subsection{Vorteile von KI für den Patienten \small{(Miriam)}}\label{sec:ki-advantages}
				Künstliche Intelligenz hilft nicht nur Ärzten, sondern auch Patienten[chall]. In den letzten Jahren ist insbesondere die Beliebtheit von Telemedizin gestiegen. Solche Anwendungen nutzen unterschiedliche Algorithmen um z. B. Informationen von tragbaren Sensoren auf Fitnesstrackern zu sammeln.[chall1] Diese Geräte werden meistens am Handgelenk getragen und versprechen die Optimierung von Wohlbefinden und Gesundheitszustand in der modernen, von Stress geplagten Gesellschaft. Der Benutzer kann beispielsweise seinen Schlafrhythmus beobachten und auswerten lassen oder seine Fitnesszustand anhand von Parametern wie Herzfrequenz oder verbrauchte Kalorien ständig überwachen.[ochallofAIinhc1] Bereits Kinder tragen gerne solche Geräte und nutzen entsprechende Anwendungen um aus Neugier z. B. die Anzahl der täglichen Schritte zu überprüfen. \\

Andere Softwareanwendungen interagieren mit dem Patienten, um eine Verdachtsdiagnose zu stellen.[ochallofAIinhc1] Solche KI - Anwendungen stellen dem Nutzer Fragen über eine vermutete Erkrankung oder ein gesundheitliches Problem und nutzen Spracherkennung oder Texterkennung, um ihm eine mündliche oder schriftliche Antwort zu ermöglichen.
Zu den beliebtesten Anwendungen dieser Art gehören ADA und Your.MG. ADA als Gesundheitshilfe wurde von einem deutschen Startup Unternehmen entwickelt und weltweit in über 130 Länder in 2016 eingeführt. [ochallofAIinhc1] Es kann kostenlos auf jede beliebige mobile Plattform heruntergeladen werden. Nach Installation und Anmeldung gibt der Nutzer seine Beschwerden und Symptome in einen Textdialog ein und bekommt schließlich Erläuterungen und Ratschläge, was er aufgrund seiner Symptome als nächstes unternehmen soll.\\
 
Eine ganz andere  Anwendung von KI ist Telehomecare, eine Alternative zum traditionellen Krankenhausaufenthalt.[cha14] Die häusliche Pflege ist einer der am schnellsten wachsenden Märkte der Welt.[cha14] In der häuslichen Pflege geht es um Patienten, die postoperativ nach einem Krankenhausaufenthalt oder aufgrund von Alter oder Gebrechlichkeit zu Hause gepflegt werden müssen. Die häusliche Pflege hat sich bereits so weit entwickelt, dass auch Palliativpatienten (sterbende Patienten) zu Hause gepflegt werden können. Dies minimiert Infektionsrisiken und senkt die Kosten für die Patienten erheblich. Zudem trägt die Pflege in einer bekannten Umgebung und mit intensivem Kontakt zur Familie stark zu positiven Behandlungsergebnissen bei.
Bei der Umsetzung von Telehomecare senden Warnsysteme, die sich in der Wohnung der Patienten befinden, Benachrichtigungen an ein Telemonitoring Zentrum, nachdem sie Daten über die spezifischen Symptome und Anzeichen des Patienten gesammelt haben.[cha14]. Diese Information werden in den Telemonitoring Zentren durch eine Krankenschwester und einen Arzt ausgewertet. Sie können in Echtzeit auf kritische Daten des Patienten zugreifen, um Komplikationen zu vermeiden, eine Verschlechterung des Zustands des Patienten zu verhindern oder auf ein Notfall zu reagieren.
Die starke Weiterentwicklung und das Wachstum der mobilen Technologien trägt dazu bei, dass die positiven Auswirkungen der KI gestützten häuslichen Pflege (Telehomecare) noch größer sind, wenn sie mit der Verwendung von tragbaren Geräten (Telemedizin) verbunden sind. Auch hier trägt der Patient Sensoren. z.B. auf einem Armband, die spezifische Information sammeln und auf die Smartphones der Überwachungszentren übertragen. [14] Der Patient kann in Echtzeit seine Herzfrequenz überwachen. Im Fall einer psychischen Krankheit hat die Familie die Möglichkeit, den Aufenthaltsort des Betroffenen zu überwachen.\\ 

Roboter in der Pflege interagieren unmittelbar mit den Patienten. Beispielsweise werden Roboter als Begleiter für die ältere Bevölkerung mit stark eingeschränkten kognitiven Fähigkeiten eingesetzt. Die Vorreiter in diesem Bereich sind die Japaner mit ihren Carebots.
			\subsection{Grenzen von KI in der Medizin  \small{(Miriam)}}\label{sec:ki-limitations}
				Trotz der großen Menge an Vorteilen, welche die KI Technologien im Gesundheitssystem mitbringen, sind die Herausforderungen für die weitere Etablierung mindestens genau so groß. Für das ordnungsgemäße Funktionieren der KI Systeme ist wesentlich, dass die Leistungserbringer im Gesundheitssystem die Daten, die zur Schulung der KI Systeme verwendet werden, auswerten, um das Einschleichen von Verzerrungen zu verhindern \cite{Chapter_14}. In der Diagnostik können diese Verzerrungen zur falschen Befunden führen.\\

Die Überanpassung von Algorithmen ist eine der Ursachen, die die Anwendung der KI in der Medizin eingrenzen. Von einer Überanpassung spricht man, wenn KI Algorithmen, die auf einen bestimmten Datensatz trainiert wurden, nur begrenzt auf andere Datensätze anwendbar sind \cite{AI_where_are_we_now}. Der KI Algorithmus hat dann ausschließlich die statischen Variationen der Trainingsdaten gelernt, anstatt der allgemeinen, zur Problemlösung notwendigen Konzepte.\\ Die Überanpassung eines Algorithmus wird von mehrere Faktoren beeinflusst wie der Größe des Datensatzes, dem Ausmaß der Heterogenität\footfullcite{heterogen} der Daten und der Verteilung der Daten innerhalb des Datensatzes. Z.B. kann ein Modell über angepasst sein, wenn sich die Krankheitshäufigkeit und die Anzahl der neu auftretenden Krankheitsfälle zwischen den Trainigs- und Testsätzen erheblich unterscheiden \cite{AI_where_are_we_now}. Überanpassung kann aber auch auftreten, wenn die Trainings- und Testsätze mit wesentlich unterschiedlichen Parametern oder Geräten erzielt wurden \cite{AI_where_are_we_now}.\\

Einer der größten Kritikpunkte an der Ki-Integration in die Medizin ist, dass sich KI Techniken von außen betrachtet wie eine Black-Box verhalten\cite{The_missing_pieces}. Gemeint ist, dass es nicht möglich ist, nachzuvollziehen, wie Deep Learning Algorithmen zu ihrer Entscheidung gekommen sind \cite{The_missing_pieces}. Zum Beispiel kann ein Arzt, der von einem KI System einen radiologischen Befund gemeldet bekommt, weder sagen, welche Verfahrensmerkmale für die Analyse verwendet wurden, noch wie sie analysiert wurden und warum der Algorithmus zu diesem Ergebnis kam\cite{The_missing_pieces}. Der Mangel an Transparenz hat deshalb die KI in der wissenschaftlichen Gemeinschaft bisher zurückgedrängt, denn oft ist das „Warum“ hinter einer Vorhersage genauso wichtig wie die Vorhersage selbst \cite{The_missing_pieces}.\\

Weitere Problemfelder für die KI im Gesundheitswesen sind Privatsphäre und Informationssicherheit.\cite{Opportunities_challenges_ai_hc}. In jedem Bereich unserer Gesellschaft ist die IT–Sicherheit inzwischen ein wichtiges Thema geworden. Im Gesundheitswesen wird dies besonders kritisch gesehen, weil Softwareanwendungen unmittelbar mit Menschenleben verbunden sind und ein Cyberangriff im ungünstigsten Fall sogar zum Tod von Patienten führen kann \cite{Opportunities_challenges_ai_hc}.\\
Auf dieser Grundlage stellen sich eine Reihe von weiteren Fragen: Welcher Schutz gilt als zuverlässig? Wer bewertet die Zuverlässigkeit? Und nicht zu vergessen: Wer träg die Verantwortung? 
Die Frage der Verantwortlichkeit wird gerade in einem konkreten Fall diskutiert: bis heute wurde die Roboterchirurgie mit 144 Todesfällen in den Vereinigten Staaten in Verbindung gebracht \cite{Chapter_14}. Wer ist verantwortlich? Das Unternehmen, der Entwickler oder der Dienstleister \cite{Chapter_14}? Diese, noch offene Fragen fordern in unserem Rechtssystem, Datenschutz und Arbeitsverträgen ein Umdenken und Änderungen. \\

	\newpage
	\section{Diskussion}\label{sec:discussion}
		\input{partials/discussion}
		\subsection{Lösungsansätze zur Zulassung von KI  \small{(Miriam)}} \label{sec:solutions}
			\begin{itemize}
	\item Unsere Vorschläge zur Lösung von Problemen, welche wir gefunden haben
\end{itemize}
		\subsection{Aktuelle Richtlinien für medizinische Produkte mit KI \small{(Alex)}}\label{sec:guidlines}
			Die aktuellen Richtlinien behandeln KI unterstützte Software zunächst als gewöhnliche Software. KI unterstützte Medizinische Software wird wie eine gewöhnliche Software zugelassen trotz des Black-Box Problems. Da die Richtlinien für Software auch für KI verwendet werden bedeutet dass das KI zuerst nur unter Laborbedingungen getestet wird, mit einem beschränktem Testdatensatz. Dies kann nicht garantieren das die KI immer funktionieren wird. Tatsache ist aber auch das die Personen in den Ämtern sich dieses Problems bewusst sind und nach besten Wissen und Gewissen versuchen werden die Sicherheit zu gewährleisten. Der amerikanische "De Novo" Weg, an den sich auch die Europäische Union orientiert, bietet die Möglichkeit weniger kritische KI im Feld zu testen und somit ein Qualitätsniveau über Zeit aufzubauen. Kritische KI, welche zum Beispiel hauptverantwortlich für das Leben eines Patienten ist kann immer noch mit den strengsten Kriterien und klinischen Test geprüft werden, bevor die allgemeine Nutzung erlaubt wird. Die aktuellen Richtlinien besitzen noch keine Spezialisierung auf KI, erlauben uns aber medizinische Produkte mit KI zu verwenden und Erfahrungen zu sammeln um ein für KI angepasstes Verfahren zu entwickeln, ohne dabei Menschenleben zu gefährden.
		\subsection{Evaluation der Zulassungskriterien für medizinische Produkte mit KI \small{(Evelyn)}}\label{sec:sufficient-criteria}
			Wie in Kapitel~\ref{sec:europe-with-ai} und Kapitel~\ref{sec:us-with-ai} zu sehen,
gibt es noch kein spezielles/spezifisches Gesetz für die Verwendung von Medizinprodukten mit KI.
Die Zulassungsverfahren von Medizinprodukten ohne KI werden bislang für die Zulassung von Medizinprodukten mit KI angewendet.
Für den Einsatz von Medizinprodukten mit KI, in kritischen Bereichen, müssen die Richtlinien angepasst und ergänzt werden.
Daraus entstehen neue Herausforderungen, die zu bewältigen sind.
Kapitel~\ref{sec:europe-with-ai} zeigt außerdem, dass von der EU-Kommission weitere Anforderungen aufgestellt werden.
In den USA werden ebenso zusätzliche Richtlinien aufgestellt.
Themen, wie ethische und rechtliche Regelungen sowie einige Fragen bleiben allerdings noch offen.
Werden die Daten der Patienten richtig verwendet? Kann die KI voreingenommen sein?
Werden Arbeitsplätze der Ärzte gestrichen? 
Wird die Versorgung auf Grund der KI besser oder schlechter? 
Mit gut durchdachten und klar definierten Regeln für die Verwendung von Medizinprodukten mit KI könnten einige
der aufkommenden Fragen und Bedenken geklärt werden.\\
Des Weiteren soll die Transparenz verbessert und exakte Qualitätsanforderungen aufgestellt werden.
Die oben genannten neuen Anforderungen und die Antworten auf die gestellten Fragen sind wichtig, 
um die Zulassungsverfahren für die Verwendung von Medizinprodukten mit KI zu erweitern,
auszubauen und präziser zu spezifizieren.
		
	
	\newpage
	\section{Fazit \small{(Miriam)}}\label{sec:conclusion}
		\input{partials/conclusion}
		\subsection{Zusammenfassung und Bewertung der Ergebnisse}\label{sec:summary}
			Der Einsatz von KI in Medizinprodukten stellt die Zulassungsbehörden vor mehrere Herausforderungen. Für das Black-Box-Problem sowie die Überanpassung von KI Algorithmen gibt es bereits Lösungsansätze, welche die Transparenz der Algorithmen verbessern. Diese ist für die Behörden von enormen Wichtigkeit, da es für KI gestützte Software keine spezifischen Zulassungsverfahren gibt. Die Software wird unter Laborbedingungen mit beschränktem Datensatz getestet. Ein erweitertes Zulassungsverfahren unter strengsten Kriterien inklusive klinischer Tests wird nur für Medizinprodukte der Klasse III durchlaufen, da diese bei Nichteinhaltung der Zulassungsanforderungen ein hohes Gefährdungspotential für Patienten besitzen.
Da die Entwicklung von KI Systemen für die Medizin stetig voranschreitet, muss die Gesetzgebung die Zulassungsverfahren fortwährend daran anpassen. Bei der Evaluation der neuen Zulassungskriterien ist es notwendig, dass auch ethische und rechtliche Regelungen in Betracht gezogen werden. Auch Datenschutz spielt bei neuen Zulassungsverfahren eine wichtige Rolle. 
In Europa gibt es aber bei neuen Zulassungskriterien für Medizinprodukte mit KI neuen Ansätze. Die EU-Kommision beschloss mit einer Expertengruppe neue, zusätzliche Anforderungen an Medizinprodukte mit KI. Sie sollen die Transparenz, Robustheit und Sicherheit der KI Anwendungen fordern. Es steht fest, dass die momentane Richtlinien und Gesetzgebung nicht ausreichen und deshalb die Zulassungsverfahren um weitere spezifische Kriterien ergänzt werden müssen.
		\subsection{Ausblick}\label{sec:perspective}
			Künstliche Intelligenz ist bereits heute aus dem Gesundheitssystem nicht mehr wegzudenken. Für die Akzeptanz der Systeme in der Bevölkerung wird die Interaktion von Mensch und KI weiterhin eine wichtige Rolle spielen und sich in Richtung einer Mensch-Computer-Symbiose weiterentwickeln. 
Denn diese Akzeptanz entscheidet am Ende, ob künstlich intelligente Medizinprodukte die Chance erhalten, ihr volles Potential zur Rettung von Leben und Verbesserung der Lebensqualität entfalten werden. 

	  

    \newpage
    \printglossary[type=\acronymtype,title={Akronyme}]
   
    \newpage
%    \section*{Literatur}
%    \addcontentsline{toc}{section}{Literatur}
    \nocite{*}
    \printbibliography[heading=bibintoc,type=article, title={Literatur}]
	
    \newpage
    \printglossary[type=main,title={Glossar}]

\end{document}
