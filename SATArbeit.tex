\documentclass[a4paper, 12pt]{article}

\usepackage{listings}

\usepackage{ulem}
\usepackage{hyperref}

\begin{document}
\title{Latex Dokument}
\author{Alex Mark Pollok,Evelyn Sophie Krebes, Miriam Lang}
\date{\today}
\maketitle

\tableofcontents

\section{Einführung (Alex)}
-grundsätzliche Zielsetzung der Untersuchung formulieren und Untersuchungsthema eingrenzen
	-Konkret zentrale Fragestellung formulieren
	-evtl. noch eine oder mehrere Hypothesen
	-Methode nennen und begründen
	- Aufbau der Arbeit Chronologisch erläutern
	-Die häufig gestellte Frage, inwieweit die genannten Punkte als zusammenhängender Text oder kapitelweise aufbereitet sein sollten, kann nicht richtig oder falsch beantwortet werden: Die Aufbereitung muss sich immer nach der jeweiligen Untersuchung richten.
	-Optional: Verweise auf die Verwendeten Quellen, beispielsweise die starke Konzentration auf Online-Quellen aufgrund fehlender anderer Quellen.

\subsection{Motivation, Kontext und Gegenstand}
\subsubsection{Motivation}
Ki werden vermutlich nicht richtig reguliert, bzw. es ist unklar wie sie reguliert werden sollen
\subsubsection{Kontext und Gegenstand}
Fokus der Arbeit auf Richtlinien und nicht Software Qualitätsmerkmale
\subsection{Ziele}
\subsubsection{Zentrale Fragestellung}
\subsubsection{Hypothesen, Unterforschungsfragen (?)}

\subsection{Vorgehensweise und Gliederung}
\subsubsection{Methodik}
Literatur Review, Research, Review of Guidelines, Comparison to Industry standards
\subsubsection{Gliederung}
Unser Vorgehen bei der Untersuchung beschreiben


\section{Darstellung des bisherigen Zulassungsverfahrens}
Zeigen wie Geräte verschiedenster Art bisher zugelassen werden
\subsection{Zulassungsverfahren in der Europäischen Union (Evelyn)}
\subsubsection{Für Medizinische Geräte ohne Künstliche Intelligenz}
Generelle Richtlinien die allgemeingültig sind
\subsubsection{Besonderheiten bei Software (?)}
Zum Beispiel Qualitätsprüfverfahren erläutern
\subsubsection{Für Medizinische Geräte mit Künstlicher Intelligenz}
Wie bei Software, aber auf KI bezogen
\subsection{Zulassungsverfahren in den US (Alex)}
\subsubsection{Für Medizinische Geräte ohne Künstliche Intelligenz}
\subsubsection{Besonderheiten bei Software (?)}
\subsubsection{Für Medizinische Geräte mit Künstlicher Intelligenz}


\section{Analyse und Bewertung}
In diesem Abschnitt werden die Forschungsfragen beantwortet
\subsection{Analyse}
\subsubsection{Besondere Kriterien für die Zulassung von medizinischen Produkten mit KI (Evelyn)}
\subsubsection{Medizin Produkte mit KI in der heutigen Medizin.(Miriam)}
\subsubsection{Welche Vorteile bringt die KI dem Patienten gegenüber? (Miriam)}
\subsection{Wo liegen die Grenzen von KI in der Medizin? (Miriam)}



\section{Fazit}
\subsection{Zusammenfassung der Ergebnisse}
\subsection{Bewertung der Ergebnisse}
\subsection{Lösungsansätze zur Zulassung von KI}
Unsere Vorschläge zur Lösung von Problemen, welche wir gefunden haben
\subsection{Mögliche Alternativen für die optimale Verwendung der KI}






\end{document}